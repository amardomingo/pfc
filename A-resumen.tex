\cleardoublepage
\phantomsection
\chapter*{Resumen}
\addcontentsline{toc}{chapter}{Resumen}

Este proyecto se ha centrado en el diseño y la implementación de un asistente personal integrado con un sistema de indexación semántica de la información, permitiendo la interacción con el usuario mediante el empleo de lenguaje natural.

Para ello, hemos analizado las tecnologías actuales que nos permiten analizar el lenguaje, recuperar e indexar información semántica y presentársela al usuario.

Así pues, hemos propuesto una arquitectura para nuestro sistema que permitiese llevar a cabo las tareas deseadas, integrando un sistema de pregunta-respuesta, un agente de conversación con la capacidad de procesar el lenguaje natural, y un sistema de recuperación de la información junto con un un modulo de indexado de dicha información.

Hemos desarrollado varios prototipos para probar nuestra arquitectura, uno de ellos centrándose en un ejemplo sencillo de apoyo a la enseñanza, proporcionando una plataforma para resolver dudas sobre el lenguaje de programación Java, respondiendo las preguntas de los alumnos en castellano, y sugiriendo nuevos temas para que los alumnos profundicen en su estudio.

Otro de los prototipos ha analizado la implementación de un sistema similar para un conjunto heterogéneo de documentos en inglés, poniendo a prueba la capacidad del alumno para diseñar un sistema modular y fácilmente ampliable.

Finalmente, hemos analizado el primer prototipo en un entorno real, recogiendo información sobre la eficiencia del sistema, la tasa de aciertos que presenta ante un corpus de preguntas, a la vez que proponíamos y realizábamos un experimento con usuarios comparando el sistema con interfaces de pregunta-respuesta tradicionales, analizando la mejora en los resultados, la experiencia de uso, y los patrones de comportamiento de los usuarios cuando se enfrentaban a un sistema como el nuestro por primera vez.

\vfill
\textbf{Palabras clave:} Tecnologías semánticas, Linked data,recuperación de la información, scrapy, scrappy, ChatScript, Solr, WSGI, asistente personal.