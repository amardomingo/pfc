\synctex=1
\documentclass[a4paper,11pt,svgnames]{book}

\usepackage[utf8x]{inputenc}
\usepackage{mathtools}
\usepackage{thesis}
\usepackage{calc}
\usepackage{enumitem}
\setlength{\marginparwidth}{2cm}
\setlength{\voffset}{-0.25in}
\setlength{\parskip}{0.1in}

\usepackage[notbib,nottoc]{tocbibind}

%\includeonly{5-architecture}
%\includeonly{6-use-case}

%\usepackage{draftwatermark}
%\SetWatermarkScale{4}
\usepackage{subfig}
 \usepackage{listings}
  \usepackage{courier}
 \lstset{
 	 basicstyle=\footnotesize\ttfamily,
         numbers=none,               
         numberstyle=\tiny,          
         %stepnumber=2,              
         numbersep=5pt,              
         tabsize=2,
         extendedchars=true,         
         breaklines=true,            
         keywordstyle=\color{red},
      	 frame=single,         
         stringstyle=\color{white}\ttfamily, 
         showspaces=false,           
         showtabs=false,             
         belowcaptionskip=5pt, 
         xleftmargin=2em, 
         xrightmargin=0.8em,
         showstringspaces=false 
 }
 \lstloadlanguages{% Check Dokumentation for further languages ...
        Python,
         Java
 }

\usepackage{caption}

\DeclareCaptionFont{white}{\color{white}}
\DeclareCaptionFormat{listing}{\colorbox{gray}{\parbox{\textwidth-20pt}{#1#2#3}}\vspace{0.01cm}}
\captionsetup[lstlisting]{format=listing,labelfont=white,textfont=white}

\captionsetup[lstlisting]{format=listing,labelfont=white,textfont=white, singlelinecheck=false, margin=0pt, font={bf,footnotesize}}


\usepackage{url}
\usepackage{tabularx}
\usepackage{graphicx}
%\usepackage[usenames,dvipsnames,table]{xcolor}
\usepackage{xcolor}
\usepackage{verbatim}
\usepackage[section]{placeins}
\usepackage{hhline}
\usepackage{multirow}

% Define the acronyms
\usepackage{acronym}

%acronym
\acrodef{BDI}{Belief Desire Intention}
\acrodef{BB}{Belief Base}
\acrodef{BML}{Behavior Markup Language}
\acrodef{ECA}{Embodied Conversational Agents}
\acrodef{IE}{Information Extraction}
\acrodef{ITS}{Intelligent Tutoring System}
\acrodef{IR}{Information Retrieval}
\acrodef{JLOO}{Java Learning Objects Ontology}
\acrodef{KB}{Knowledge-Base}
\acrodef{LOD}{Linked Open Data}
\acrodef{NL}{Natural Language}
\acrodef{NLI}{Natural Language Interface}
\acrodef{NLP}{Natural Language Processing}
\acrodef{NLU}{Natural Language Understanding}
\acrodef{MAS}{Multi Agent System}
\acrodef{PoS}{Part of Speech}
\acrodef{QA}{Question Answering}
\acrodef{SKOS}{Simple Knowledge Organization System}
\acrodef{SPARQL}{SPARQL Protocol and RDF Query Language}
\acrodef{OWL}{Web Ontology Language}
\acrodef{SW}{Semantic Web}
\acrodef{OoB}{Out of Band}
\acrodef{eDisMax}{Extended Disjunction Max}
\acrodef{RDF}{Resource Description Framework}
\acrodef{AIML}{Artifcial Intelligence Markup Language}
\acrodef{LMF}{Linked Media Framework}
\acrodef{WSGI}{Web Server Gateway Interface}
\acrodef{FOAF}{Friend of a Friend}
\acrodef{SIOC}{Semantically-Interlinked Online Communities}

% Reset the acronyms at the beginning of each chapter
\usepackage{etoolbox}
\preto\section\acresetall

\usepackage[pdftex,
            pdfauthor={Alberto Mardomingo Mardomingo},
            pdftitle={Design of a personal agent integrated with a Linked Data Indexing System},
            pdfsubject={Master Thesis},
            pdfkeywords={Semantic, Linked Data, RDF, Scrapping, natural language},
            pdfproducer={PDFTex},
            colorlinks=true,linkcolor=black,citecolor=black,urlcolor=black,hypertexnames=false]{hyperref}

\setcounter{secnumdepth}{3}

\usepackage{framed}
\usepackage{fixltx2e}

\colorlet{punct}{red!60!black}
\definecolor{background}{HTML}{EEEEEE}
\definecolor{delim}{RGB}{20,105,176}
\colorlet{numb}{magenta!60!black}

\usepackage{listings}
\lstset{escapeinside=||}


% switch used as state variable
\newif\ifcolonfoundonthisline

\lstdefinelanguage{JavaScript}{
  keywords={typeof, new, true, false, catch, function, return, null, catch, switch, var, if, in, while, do, else, case, break},
  keywordstyle=\color{blue}\bfseries,
  ndkeywords={class, export, boolean, throw, implements, import, this},
  ndkeywordstyle=\color{darkgray}\bfseries,
  identifierstyle=\color{black},
  sensitive=false,
  comment=[l]{//},
  morecomment=[s]{/*}{*/},
  commentstyle=\color{purple}\ttfamily,
  stringstyle=\color{red}\ttfamily,
  captionpos=b,
  morestring=[b]',
  morestring=[b]",
  numbers=left,
  frame=lines,
  backgroundcolor=\color{background}
}
\lstdefinelanguage{json}{
    basicstyle=\normalfont\ttfamily,
    numbers=left,
    numberstyle=\scriptsize,
    stepnumber=1,
    numbersep=8pt,
    captionpos=b,
    showstringspaces=false,
    breaklines=true,
    frame=lines,
    backgroundcolor=\color{background},
    literate=
     *{0}{{{\color{numb}0}}}{1}
      {1}{{{\color{numb}1}}}{1}
      {2}{{{\color{numb}2}}}{1}
      {3}{{{\color{numb}3}}}{1}
      {4}{{{\color{numb}4}}}{1}
      {5}{{{\color{numb}5}}}{1}
      {6}{{{\color{numb}6}}}{1}
      {7}{{{\color{numb}7}}}{1}
      {8}{{{\color{numb}8}}}{1}
      {9}{{{\color{numb}9}}}{1}
      {:}{{{\color{punct}{:}}}}{1}
      {,}{{{\color{punct}{,}}}}{1}
      {\{}{{{\color{delim}{\{}}}}{1}
      {\}}{{{\color{delim}{\}}}}}{1}
      {[}{{{\color{delim}{[}}}}{1}
      {]}{{{\color{delim}{]}}}}{1},
}

\definecolor{dkgreen}{rgb}{0,0.6,0}
\definecolor{gray}{rgb}{0.5,0.5,0.5}
\definecolor{mauve}{rgb}{0.58,0,0.82}
\definecolor{gray}{rgb}{0.4,0.4,0.4}
\definecolor{darkblue}{rgb}{0.0,0.0,0.6}
\definecolor{lightblue}{rgb}{0.0,0.0,0.9}
\definecolor{cyan}{rgb}{0.0,0.6,0.6}
\definecolor{darkred}{rgb}{0.6,0.0,0.0} 
\lstdefinelanguage{XML}{
  basicstyle=\ttfamily\footnotesize,
  columns=fullflexible,
  showstringspaces=false,
  numbers=left, % where to put the line-numbers
  numberstyle=\tiny\color{gray}, % the style that is used for the line-numbers
  stepnumber=1,
  numbersep=5pt, % how far the line-numbers are from the code
  backgroundcolor=\color{background}, % choose the background color. You must add \usepackage{color}
  showspaces=false, % show spaces adding particular underscores
  showstringspaces=false, % underline spaces within strings
  showtabs=false, % show tabs within strings adding particular underscores
  frame=line, % adds a frame around the code
  rulecolor=\color{black}, % if not set, the frame-color may be changed on line-breaks within not-black text (e.g. commens (green here))
  tabsize=2, % sets default tabsize to 2 spaces
  captionpos=b, % sets the caption-position to bottom
  breaklines=true, % sets automatic line breaking
  breakatwhitespace=false, % sets if automatic breaks should only happen at whitespace
  title=\lstname, % show the filename of files included with \lstinputlisting;
  % also try caption instead of title
  commentstyle=\color{gray}\upshape
  morestring=[s][\color{mauve}]{"}{"},
  morestring=[s][\color{black}]{>}{<},
  morecomment=[s]{<?}{?>},
  morecomment=[s][\color{dkgreen}]{<!--}{-->},
  stringstyle=\color{black},
  identifierstyle=\color{lightblue},
  keywordstyle=\color{red},
  morekeywords={xmlns,xsi,field,type,fieldType,copyField,analyzer,tokenizer,filter,foaf,rdf,sc}% list your attributes here
} 


\lstdefinestyle{listXML}{language=XML, basicstyle=\ttfamily\diny, extendedchars=true,  belowcaptionskip=5pt,xleftmargin=0.4em, xrightmargin=0.3em, numbers=none, frame=single, breaklines=true, breakatwhitespace=true, breakindent=0pt, emph={}, emphstyle=\color{red}, basicstyle=\small\ttfamily, columns=fullflexible, showstringspaces=false, commentstyle=\color{gray}\upshape,
morestring=[b]",
morecomment=[s]{<?}{?>},
morecomment=[s][\color{orange}]{<!--}{-->},
keywordstyle=\color{cyan},
stringstyle=\color{black},
tagstyle=\color{darkblue},
morekeywords={xmlns, version, type, field, copyField, fieldType}
}



\lstdefinestyle{mono}{
   framesep=8px,
   extendedchars=true,
   basicstyle=\ttfamily,
   showstringspaces=false,
   showspaces=false,
   tabsize=2,
   breaklines=true,
   showtabs=false,
   xleftmargin=8pt,
   xrightmargin=8pt
}


\lstdefinestyle{commands}{
   framesep=8px,
   extendedchars=true,
   basicstyle=\ttfamily,
   showstringspaces=false,
   showspaces=false,
   tabsize=2,
   breaklines=true,
   showtabs=false,
   xleftmargin=8pt,
   xrightmargin=8pt
}

\lstdefinestyle{consola}
   {basicstyle=\scriptsize\ttfamily,
    backgroundcolor=\color{white},
    frame=lrtb,
    numbers=none,
    xleftmargin=4pt,
    xrightmargin=4pt
   }


\definecolor{chapterdetails}{HTML}{6DBAFF}

\usepackage[sf,bf]{titlesec}
\titleformat{\chapter}[display]
  {\normalfont\Large\sffamily\raggedleft}
  {\vspace{5cm}\MakeUppercase{\chaptertitlename}%
    \rlap{ \resizebox{!}{1.5cm}{\thechapter} \color{chapterdetails}\rule{5cm}{1.5cm}}}
  {10pt}{\Huge}[{\color{chapterdetails}\titlerule[0.8mm] }]
\titlespacing*{\chapter}{0pt}{30pt}{20pt}

\newenvironment{chapterintro}
{% This is the begin code
\large\it
}
{% This is the end code
}


% Tick symbols
\newcommand{\tickYes}{\checkmark}
\newcommand{\tickNo}{\hspace{1pt}\ding{55}}
% Fancy header
\usepackage{fancyhdr}
%Fancy chapter cover style


% Fancy box
\usepackage{fancybox} 
\setlength{\fboxrule}{1 pt} \setlength{\fboxsep}{10pt} \setlength{\shadowsize}{3pt}

\begin{document}
\newcommand\litem[1]{\item{\bfseries #1 }}
\renewcommand{\arraystretch}{1.5} %Makes tables less crammed

\newcommand\headcell[1]{%
  \multicolumn{1}{|c|}{\cellcolor{DodgerBlue}\bfseries\sffamily\textcolor{white}{#1}}
}

\pagenumbering{Roman}

\cleardoublepage
\thispagestyle{empty}
\vspace*{3\baselineskip}
{\large{\bf PROYECTO FIN DE CARRERA}}
\vspace{0.5cm}

\begin{rm}
\begin{tabular}{p{3cm}p{10cm}}
\textbf{Título:} & Desarrollo de un Asistente Personal integrado con un Sistema de Indexación Semántica de Información\\ 
\textbf{Título (inglés):} & Design of a personal agent integrated with a Linked Data Indexing System\\ 
\textbf{Autor:} & Alberto Mardomingo Mardomingo \\ 
\textbf{Tutor:} & Carlos A. Iglesias Fernández\\ 
\textbf{Departamento:} & Ingeniería de Sistemas Telemáticos \\ 
\end{tabular} \end{rm} \vspace{1cm}

{\large{\bf MIEMBROS DEL TRIBUNAL CALIFICADOR}} \vspace{0.5cm}

\begin{rm}
\begin{tabular}{p{3cm}p{10cm}}
\textbf{Presidente:} & Mercedes Garijo Ayestarán\\
\textbf{Vocal:} & José Carlos González\\
\textbf{Secretario:} & Carlos Ángel Iglesias Fernández\\
\textbf{Suplente:} & Tomás Robles Valladares
\end{tabular}
\end{rm}
\vspace{1cm}

{\large{\bf FECHA DE LECTURA:}}
\vspace{1cm}

{\large{\bf CALIFICACIÓN:}}
\pagestyle{empty}
\cleardoublepage
\vspace*{\baselineskip}
\begin{center}
	{\LARGE\rm\textbf{UNIVERSIDAD POLITÉCNICA DE MADRID}\\
	\vspace{1.0cm}
	 ESCUELA TÉCNICA SUPERIOR DE\\ INGENIEROS DE TELECOMUNICACIÓN
	  }  \\

	 {\Large\rm Departamento de Ingeniería de Sistemas Telemáticos\\
	 Grupo de Sistemas Inteligentes  }  \\

\begin{figure}[!htbp]
	\centering
    \includegraphics[width=0.7\textwidth]{img/logos/logo_etsit.jpg}

\end{figure}
	\vspace{1.0cm}
	{{\LARGE\rm PROYECTO FIN DE CARRERA\\
	\vspace{2.0cm}
	 \textbf{DESIGN OF A PERSONAL AGENT}\\	 
	 \textbf{INTEGRATED WITH A LINKED}\\ 
	 \vspace{0.5cm}
	 \textbf{DATA INDEXING SYSTEM} }}  \\
	 
	 \vspace{1.0cm}
     \Large\rm\textbf{Alberto Mardomingo Mardomingo}\\
	 \vspace{1.0cm}
	 Julio de 2015
\end{center}  

\cleardoublepage

\begin{tabular}{p{10cm}p{4cm}}
&\\
&\\
&\\
&\\
&\\
&\\
&\\
&\\
&\\
&\emph{42}\\
&\\
\end{tabular}

\cleardoublepage
\phantomsection
\chapter*{Resumen}
\addcontentsline{toc}{chapter}{Resumen}

Este proyecto se ha centrado en el diseño y la implementación de un asistente personal integrado con un sistema de indexación semántica de la información, permitiendo la interacción con el usuario mediante el empleo de lenguaje natural.

Para ello, hemos analizado las tecnologías actuales que nos permiten analizar el lenguaje, recuperar e indexar información semántica y presentársela al usuario.

Así pues, hemos propuesto una arquitectura para nuestro sistema que permitiese llevar a cabo las tareas deseadas, integrando un sistema de pregunta-respuesta, un agente de conversación con la capacidad de procesar el lenguaje natural, y un sistema de recuperación de la información junto con un un modulo de indexado de dicha información.

Hemos desarrollado varios prototipos para probar nuestra arquitectura, uno de ellos centrándose en un ejemplo sencillo de apoyo a la enseñanza, proporcionando una plataforma para resolver dudas sobre el lenguaje de programación Java, respondiendo las preguntas de los alumnos en castellano, y sugiriendo nuevos temas para que los alumnos profundicen en su estudio.

Otro de los prototipos ha analizado la implementación de un sistema similar para un conjunto heterogéneo de documentos en inglés, poniendo a prueba la capacidad del alumno para diseñar un sistema modular y fácilmente ampliable.

Finalmente, hemos analizado el primer prototipo en un entorno real, recogiendo información sobre la eficiencia del sistema, la tasa de aciertos que presenta ante un corpus de preguntas, a la vez que proponíamos y realizábamos un experimento con usuarios comparando el sistema con interfaces de pregunta-respuesta tradicionales, analizando la mejora en los resultados, la experiencia de uso, y los patrones de comportamiento de los usuarios cuando se enfrentaban a un sistema como el nuestro por primera vez.

\vfill
\textbf{Palabras clave:} Tecnologías semánticas, Linked data,recuperación de la información, scrapy, scrappy, ChatScript, Solr, WSGI, asistente personal.
\cleardoublepage
\phantomsection
\chapter*{Abstract}
\addcontentsline{toc}{chapter}{Abstract}

This project has focused on the design and implementation of a Personal Agent integrated with a Linked Data System, allowing users to interact with it using natural language.

In order to achieve our goal, we have studied current technologies that would allow us to analyse natural language, retrieve and index semantic information, and present it to the user.

We have therefore proposed an architecture for our system that would allow us to perform these tasks, integrating a question answering system, a conversational agent with natural language processing capabilities, as well as an information retrieval system, along with a linked data indexing system.

Based on this architecture, we developed several prototypes that would allow us to test the architecture. The first one of them focus on a simple example of e-learning in Spanish, providing students with a platform to solve their questions about the Java programming language, and offering new topics so they can delve into their study.

A different prototype has analysed the implementation of a similar system for a heterogeneous collection of documents in English, testing the student ability to design a modular system that can be easily extended.

Finally, we have studied our first prototype in a real scenario, gathering information about its efficiency, the success rate while tested with a corpus, as well as proposing an experiment with users. The experiment compared the system with traditional question answering systems, analysing the improvement in the results, the user experience, as well as the patterns in user behaviour when faced with a system like ours for the first time.

\vfill
\textbf{Keyword:} Semantic technologies, Linked data, Information retrieval, scrapy, scrappy, ChatScript, Solr, WSGI, Personal Assistant
\include{C-acknowledgment}
\tableofcontents

\listoffigures

\listoftables

\lstlistoflistings

\cleardoublepage

\chapter*{Acronyms}
\addcontentsline{toc}{chapter}{Acronyms}
\begin{acronym}

\acro{BDI}{Belief Desire Intention}
\acro{BB}{Belief Base}
\acro{BML}{Behavior Markup Language}
\acro{ECA}{Embodied Conversational Agents}
\acro{IE}{Information Extraction}
\acro{ITS}{Intelligent Tutoring System}
\acro{IR}{Information Retrieval}
\acro{JLOO}{Java Learning Objects Ontology}
\acro{KB}{Knowledge-Base}
\acro{LOD}{Linked Open Data}
\acro{NL}{Natural Language}
\acro{NLI}{Natural Language Interface}
\acro{NLP}{Natural Language Processing}
\acro{NLU}{Natural Language Understanding}
\acro{MAS}{Multi Agent System}
\acro{PoS}{Part of Speech}
\acro{QA}{Question Answering}
\acro{SKOS}{Simple Knowledge Organization System}
\acro{SPARQL}{SPARQL Protocol and RDF Query Language}
\acro{OWL}{Web Ontology Language}
\acro{SW}{Semantic Web}
\acro{OoB}{Out of Band}
\acro{eDisMax}{Extended Disjunction Max}
\acro{RDF}{Resource Description Framework}
\acro{AIML}{Artifcial Intelligence Markup Language}
\acro{LMF}{Linked Media Framework}
\acro{WSGI}{Web Server Gateway Interface}
\acro{FOAF}{Friend of a Friend}
\acro{SIOC}{Semantically-Interlinked Online Communities}
\end{acronym}
\cleardoublepage



%Header style
\pagestyle{fancy}
\fancyhf{}
\fancyhead[RO]{\sffamily \slshape \rightmark}
\fancyhead[LE]{\sffamily \slshape \leftmark}
%\renewcommand{\footrulewidth}{0.4pt} % grosor de la línea del pie
\fancyfoot[OR,EL]{\rmfamily \thepage} % texto derecha del pie
\pagenumbering{arabic}

\chapter{Introduction}
\label{chapter:intro}

\begin{chapterintro}

In this chapter we will introduce the objectives of this Master Thesis as well as the motivation for them, and describe the structure of this document.
 
\end{chapterintro}

\cleardoublepage

\section{Context}

Personal agents are present in many fields, from educative platforms~\cite{fonte2012intelligent} to virtual city tours~\cite{bogdanovych2012the} or database querying~\cite{augello2009semantic}. In this document we present our project, an architecture for conversational systems over linked data, as well as two prototypes, one applied to the educative platforms field, and the other about academic information of a research group. In both cases there has been an increase in the influx of information over the last few years, in a way that made necessary for the end users to have a platform that would allow them to easily access said information.

One alternative for accessing the desired information is using conversational agents, allowing the end user to express their requests and desires in natural language, simplifying the interactions with the system. We will therefore study the technologies available for understanding natural language, as well as information retrieval and indexing technologies, allowing us to build a system that will interact with the users, both answering their questions in natural language, and proposing related topics to look into.

Understanding Natural Language involves using grammars and semantics, statistical methods or templates to identify keywords from the user's input and understand what they are requesting. Maintaining these services whilst reducing the cost and improving their knowledge presents an important challenge for researches, specially when considering than including assistance into the system also involves modelling the new actions the system will have to be able to respond to.

In order to be able to suggest new topics, we will study linked data systems. According to the W3C, the semantic web is ``\emph{a Web of Data — of dates and titles and part numbers and chemical properties and any other data one might conceive of}'', and Linked Data implies making the data available in a standard format, reachable and manageable by Semantic Web tools, as well as the \emph{relationships among data}, therefore creating a collection of interrelated datasets, that can be referred as Linked Data\footnote{\url{http://www.w3.org/standards/semanticweb/data}}.

Finally, the heterogeneous nature of systems being used by the user, makes it interesting to develop the interface for our system as a web application, allowing access from any platform with access to a web browser.

\section{Goals}

The main goal of this Master Thesis is to develop a system that would allow users to interact using Natural Language with a Linked Data System, receiving the information they have asked for, as well as suggestions and information about related topics.

We present a web application that can be used to navigate the information, ask questions and chat with the agent. The system will then be able to find the answer in the Linked Data System, a knowledge base that can be improved using information retrieval techniques.

We will analyse the state of the art systems for Natural Language Processing, in order to chose the most adequate for our system, as well as the systems capable of storing linked data, to be able to select the one that best fit our necessities.

Furthermore, we will study the implementation of our system for two different knowledge fields, the first one with a simple collection of similar documents, and the other with multiple data sources and document structures, but both of them presenting a similar interface for the end user.

Finally, we will evaluate our system, both taking measures regarding its general performance, and designing an experiment with real users to study their responses and experience with our system.

\section{Structure of the document}

In this section we will provide a brief overview of the structure of this Master Thesis. Each chapter is as follows:

\emph{Chapter \ref{chapter:intro}} provides an introduction to the project, explaining the basic concepts, as well as the goals for the project.

\emph{Chapter \ref{chap:state_of_the_art}} list all the technologies used in this project, along with some other systems that are thematically related to the one proposed in this Master Thesis.

\emph{Chapter \ref{chap:architecture}} describes the architecture proposed for our system, justifying and explaining each module and its functions.

\emph{Chapter \ref{chap:usecasejava}} shows the functionality of one of the prototypes developed for the project, a system to facilitate the search of information about the Java programming language, explaining the function of each module as well as the interactions between them.

\emph{Chapter \ref{chap:usecasegsi}} demonstrates a prototype for a different knowledge field, the information about the Intelligent Systems Group\footnote{\url{http://www.gsi.dit.upm.es/}}, its members, publications and projects. 

\emph{Chapter \ref{chap:evaluation}} discuss the results of the first prototype, showing the performance and analysing the methodology and results of a test with real users.

\emph{Chapter \ref{chap:conclusion}} analyses the overall result of the Master Thesis, summarizing the goals, and considering possible future developments.

\chapter{Enabling technologies}
\label{chap:state_of_the_art}

\begin{chapterintro}

In this chapter, a brief introduction of the state of the art for conversational agents and Question Answering is presented. Likewise, we will take a short look at some Linked Open Data systems, and the ways to recover data for them.


\end{chapterintro}

\cleardoublepage

\section{Overview}

Conversational agents, presented in section \ref{sec:conv_agents} are systems that allow an user to interact with using natural language, the same way they will interact will another humar being. This is achieved by using engines that analyse the user input, process it, and provide the best possible answer given the knowledge of the system.

Question answering systems work in a similar way, but rather than provide a response in natural language, they present the user the resource where the answer to their question is located, usually by translating the question to a specialised query for a given database.

The aforementioned database is usually a Linked Open Data System. This systems allow the publication of Semantic data, connecting it to the world and making therefore easily accesible and linkable.

Finally, we will study the way of populating the system, using web scrapping techniques in order to recover the information when it is not presented as Linked Open Data.


\section{Conversational Agents}
\label{sec:conv_agents}

In this section we will discuss the evolution of conversational engines, and discuss a few of the techniques and technologies utilized implementing them. We will provide a small overview to \ac{AIML} and some engines implemented with it, as well as other engines to finalise with a small description of ChatScript.

One of the starting points when studying conversational agents is A.L.I.C.E. an free natural language artificial intelligence chat robot that utilizes AIML for creating responses based on the user input to the system. ALICE won the 2000, 2001, and 2004 Loebner prizes, becoming a starting point while developing conversational agents. It makes use of the pattern-matching ability of AIML, with 120.000 patterns that can either trigger a response o redirect the input to another pattern. ALICE was inspired by Eliza, on the first examples of natural language processing using simple patterns, written at MIT by Joseph Weizenbaum between 1964 and 1966.

Along with ALICE, a number of other AIML conversational agents have been presented to the Loebner contest, by different authors, usually getting good results, like Mitsuku, by Steve Worswick, who won the 2013 edition of the contest, and was among the 4 finalists in 2014, three of them using AIML. Another example of an AIML bot is Izar, by Brian Rigsby, who achieved second place in the 2014 contest

The winner of the 2010, 2011 and 2014 contests was Bruce WillCox, using different chatbots, all of them written in ChatScript. Chatscript was presented in 2010, written in C++, and later released as Open Source. Whilst AIML aims to pattern-match words, ChatScript claims to match in a general meaning basis, focusing on detecting equivalence and paying heavy attention to sets of words and canonical representation, and providing a simple way of storing user data, in a machine readable format.

\subsection{\ac{AIML}}

\ac{AIML}~\cite{aimlprimer} is a widely use XML dialect for creating conversational language. It was developed between 1995 and 2002 by Richard S. Wallace and the free software community, and has remained relevant to this date, including the draft for a major upgrade, AIML 2.0, released in the early 2013, and currently being working on. The original version of AIML had seven design goals, stated in its primer:
\begin{enumerate}%[topsep=0pt,itemsep=-1ex,partopsep=1ex,parsep=1ex]
  \item Shall be easy to learn.
  \item Shall encode the miniaml concept set necessary to enable a stimulus-response knowledge system modelled on that of the original A.L.I.C.E.
  \item Shall be compatible with XML.
  \item It shall be easy to write programs that process AIML documents.
  \item \ac{AIML} objects should be human-legible and reasonably clear.
  \item The design of \ac{AIML} shall be formal and concise.
  \item \ac{AIML} shall not incorporate dependencies upon any other language.
\end{enumerate}

There are more than twenty tags for the AIML language~\cite{aliceaiml}, but the most important units are \emph{aiml}, the tag that defines a document as \ac{AIML}, \emph{category} marking a ``unit of knowledge'' in the bot's knowledge base, \emph{pattern} contains a simple pattern that will be compared with the user input, and \emph{template} containing the response to a user input.

\begin{center} 
  \begin{lstlisting}[language=XML, captionpos=b, caption=Example AIML code, label=listing:exampleaiml]   
    <category>
        <pattern> WHO ARE YOU </pattern>
        <template>I am a bot </pattern>
    <category>
 \end{lstlisting}
\end{center}
The free A.L.I.C.E. \ac{AIML}~\footnote{\url{http://www.alicebot.org/downloads/}} includes a knowledge base of 41.000 categories, and can be used as a base for others bots.

\subsubsection{\ac{AIML} 2.0}

In January 2013 the ALICE A.I. Foundation released a draft specification for a major update for \ac{AIML}~\footnote{\url{http://alicebot.blogspot.com.es/2013/01/aiml-20-draft-specification-released.html}}, aiming to provide new features while trying to keep \ac{AIML} as simple as possible. \ac{AIML} 2.0 combines Pandorabots extensions to the language, \ac{OoB} tags and a collection of new features. The full list of new features can be found in the Working Draft~\cite{aiml20draft}, some of the more relevant are:

\begin{itemize}%[topsep=0pt,itemsep=-1ex,partopsep=1ex,parsep=1ex]
  \item Zero+ wildcards, allowing to match zero or more words
  \item Setting matching priority for certain words.
  \item Loops.
  \item \ac{OoB} tags.
  \item Local variables.
\end{itemize}

\subsubsection{\ac{AIML} implementations}

\ac{AIML} has been implemented in multiple languages, including Java, Python, PHP or C++. Some of those implementations are listed in the ALICE downloads page~\footnote{\url{http://www.alicebot.org/downloads/programs.html}}, and we will briefly describe some of them bellow:

\begin{enumerate}
 \item \textbf{Program D} is a Java implementation, open source, implementing the \ac{AIML} specification. It supports multiple bots per instance, and provides multiple ways to interfaces to interact with the service, providing a J2EE release allowing deployment as a webservice. However, the development of this project has been stagnated for years, with its last release in 2006.
 \item \textbf{ChatterBean} is another Java implementation, aiming to be \ac{AIML} 1.0.1 compliant, using a JavaBeans plugin architecture, and released under a GPL license. However, the last version was released on 2006 and the development since then seems to have stopped.
 \item \textbf{Program O}, written in PHP with MySQL, is an \ac{AIML} engine with a web interface providing a number of remarkable features, including an administration panel, with configuration, teaching an testing interfaces. It stores the \ac{AIML} files in a MySQl database in order to improve its performance, and can assign different personalities to each bot. It is under active development by Ellizabeth Perreau and Dave Morton, with it latest version released in May 2014. 
\end{enumerate}


\begin{figure}[!htbp]
    \centering
    \includegraphics[width=0.9\textwidth]{img/screens/programo.png}
    \caption{Demo interface for Program-O.}
    \label{fig:programo1}
\end{figure}


\subsection{ChatScript}
\label{subsec:chatscript}

ChatScript, written in C++ by Bruce Willcox, is a chatbot engine aiming to overcome the limitations of \ac{AIML}, by providing pattern matching on general meaning rather than particular words. It was originally created for Avatar Reality, a start-up company that wanted to create a virtual world called Blue Mars~\cite{Wilcox2010suzzete}, and it was eventually released as open source as per the requirements for the Loebner prize.

Since then, ChatScript has been updating and introducing improvements, up to version 5.4~\cite{ChatScriptSourceForge}. Some of the features in ChatScript include:
\begin{itemize}
 \item Efficient, easily readable output rules
 \item Zero-length wildcards
 \item Concise pattern matching through the use of concepts
 \item Built-in data covering multiple subjects and topics
\end{itemize}

By matching in meaning, ChatScript claims to be able to provide a better user experience than \ac{AIML}, with a much smaller rule set, improving maintainability and ease to modify the bot whenever necessary. More details on ChatScript are provided in section ~\ref{subsec:chatscript}.

\subsubsection{Basic syntax, topics and rules}

In Chatscript, rules are grouped in topics, and stored in ``.top'' files, allowing the designer to map the point of the conversation, therefore making a better response for the user. And example topic file can be found in listing~\ref{listing:exampletop}. In this example, the topic will be triggered when the user writes an input containing the words ``name'', ``here'', ``what'', or any other word specified in the system dictionaries ``emogoodbye'', ``emohello'' or ``emohowzit''. Then, it will attempt to match the user input with the patterns ``what is your name'' and ``what ~be''. If no pattern matches, it will output the gambit ``Have you been here before?''.

\begin{center} 
  \begin{lstlisting}[language={}, captionpos=b, caption=Example topic file, label=listing:exampletop]   
    topic: ~INTRODUCTIONS (~emogoodbye ~emohello ~emohowzit name here what )
    
    #!x issued only once, then erased
    t: Have you been here before?
    
    #! what is your name 
    u: ( what is your name ) 
        My name is Harry.
        
    #! what are you ?
    ?: ( what ~be )
        I am a bot. Are you also a bot?
        a: (~no)
            Oh, a human... How can I help you?
  \end{lstlisting}
\end{center}

In this simple example, we find some of ChatScript syntaxes elements. We can see responders, gambits and rejoinders, as well as the use of topics and concepts. These are part of ChatScript rules structure, but there are some more elements not shown here. A slightly more comprehensive list is as follows:
\begin{itemize}
 \item \textbf{Input matching:} The pattern can start with either ``s:'', ``?:'' or ``u:'', indicating they are supposed to respond to statements, questions or both. The parenthesis then indicate the pattern itself, which can have multiple elements:
    \subitem \emph{General words:} A simple word to look for on the sentence. ChatScript handles plurals and verb conjugations.
    \subitem \emph{Concepts:} Starting with \~ , a concept is a set of words with a common general meaning.
    \subitem \emph{Wildcards:} Allowing to match any word, or using a cardinal, can specify the number of words to match with the wildcard.
    \subitem \emph{Start and end of sentence:} Using > and <, its possible to indicate whether the pattern should match the start or the end of the sentence.
    \subitem \emph{Variables: } Starting the pattern element, whether is a wildcard, a concept or a word, using ``\_'', will store the matched word in a variable for future use, be it in the response or somewhere else.
  \item \textbf{Gambit:} Lines starting with ``t:'' indicate a gambit, that will be triggered when the bot cannot find an appropriate response.
  \item \textbf{Bot Response:} Immediately after the pattern, the following lines until the next chatscript syntax element indicate how the bot response is formed.
  \item \textbf{Variables:} Not shown in the previous example, any word starting with ``\$'' is a global variable, and can be assigned a value that will be remembered for the user. If the variable starts with ``\$\$'', the variable will be local, only lasting until the end of the current answer.
  \item \textbf{Rejoinders} Starting with single letters from ``a:'' through ``q:'' to indicate nesting depth, this indicate patterns that will trigger when the user responds to a specific bot output.
  \item \textbf{Concepts:} Sets of word with a specific meaning, when referring to a concept in a pattern, it will match if any word in the set matches the input. ChatScript includes a large set of concepts by default, but the user can define its own sets.
  \item \textbf{Topics:} rules are grouped in topics. Rules inside the topic will only be tested against the input when the input itself contains any of the words specified while defining the topic.
\end{itemize}

A more detailed approach to ChatScript rules, matching and functions can be found in its documentation\footnote{\url{http://sourceforge.net/projects/chatscript/files/}}.

\subsubsection{Deploying a bot with ChatScript}

By default, ChatScript can be run in local or server mode. Local mode allows for direct interaction from the command line, using the standard input and output to communicate with the user. This mode is specially usefull for debugging the rules when developing bots, since it gives access to debugging commands build with the system.

For production environments, ChatScript can be run in server mode, where it listens it offers a TCP interface. This interface will receive messages containing the bot name, the user name and the user input, and return the bot response for the given input.

\section{Question Answering Systems}
\label{sec:qa_sys}

\ac{QA} is a discipline concerned with automatically provide answers to questions presented in natural language, using a number of different approaches in order to process the question into a query the system can understand, and, therefore, answer.

In general, we can differentiate six major general approaches~\cite{unger2014an}:

\begin{itemize}
  \item \textbf{Controlled natural languages:} The system only takes into account a well-defined subset of a given natural language that can be unambiguously interpreted.
  \item \textbf{Formal grammars processing:} Relaying on linguistics to assign syntactic and semantic representations to lexical units, as well as compositional semantics, this systems compute a representation of the question. Two examples of this systems could be ORAKEL~\cite{cimiano2008towards} and Pythia~\cite{unger2011pythia}
  \item \textbf{Mapping linguistics to semantic structures:} Systems designed under this principle rely on a measure of similarity between elements in the query and the predicates, subjects or objects in the knowledge base. PowerAqua~\cite{lopez2011poweraqua} and Aqualog~\cite{lopez2007aqualog} are two examples of this approach.
  \item \textbf{Template-based:} Taking two stages, this approach first construct a query based on the linguistic analysis of the input question, and the matchs the expressions in the question with elements from the dataset. TBSL~\cite{unger2012template} implements this approach.
  \item \textbf{Graph exploration:} This approach maps elements of the question to entities in the knowledge base, and proceeds navigation from these pivot elements to navigate the graph, seeking to connect the entities to yield a connected query. This example is taken by the TREO~\cite{freitas2011querying} system
  \item \textbf{Machine learning:} question answering has been considered a machine learning problem, with either models for joint query interpretation and response ranking, aiming at learning semantic parsers given a knowledge base and a set of questions and answers, or systems with an algorithm for matching natural language expressions and ontology concepts, as well as an algorithm for storing matches in a parse lexicon.
\end{itemize}

%\emph{\textcolor{red}{Shortly describe the exampe systems?}}

To the aforementioned systems, we have to add another one: IBM Watson's DeepQA. Designed to be a contestant in the Jeopardy! quiz show, the DeepQA system had multiple information sources, the DeepQA system focused on extracting and scoring evidence from unstructured data, although it also used structured an semantic data sources. It was build as a massively probabilistic evidence-based architecture using more than 100 different techniques for analysing natural language. It was build using Apache UIMA\footnote{\url{http://uima.apache.org/}}, an Unstructured Information Management system.

%\emph{\textcolor{red}{Add figure or list with the structure?}}

\section{Linked Data Systems}
\label{sec:linkd_sys}

Linked Data consists in a set of rules about publishing data in the web so it can be interlinked and accessed using semantic queries. The term was first used by Tim Berners-Lee while talking about the Semantic Web project. Additionally, Linked Open Data is an extension of the Linked Data concept, requiring that the data provided is open content.

\begin{figure}[!htbp]
  \begin{minipage}{\linewidth}
    \centering
    \includegraphics[width=\textwidth]{img/enabling/lod-cloud.png}
    \caption[Linked Open Data cloud]{%
      Linked Open Data cloud%
      \footnote{\url{http://lod-cloud.net/}}%
      }
    \label{fig:arch2}
  \end{minipage}
\end{figure}

There are a number of Linked Data Systems publicly available to the public, as long as multiple projects allowing to deploy your own linked data services. Some of them are described in the next sections. 

\subsection{Apache Lucene and Solr}
\label{sec:statesolr}

Not considered Linked Data Systems on its own, Apache Lucene~\footnote{\url{http://lucene.apache.org/}} is a high-performance full-featured text search engine, that can be used as the foundation of many systems. Apache Solr is built on top of Lucene, providing distributed indexing, replication and querying, with multiple features and functionalities~\footnote{\url{http://lucene.apache.org/solr/}}:

\begin{itemize}%[topsep=0pt,itemsep=-1ex,partopsep=1ex,parsep=1ex]
  \item Full-text search, with powerful matching capabilities, powered by Lucene.
  \item Optimized for High Volume traffic.
  \item Standards Based Open interfaces, including json, xml and http.
  \item Comprehensive Administration Interfaces, making it easy to handle Solr instances.
  \item Easy Monitoring, publishing relevant data via JMX
  \item Highly scalable and Fault tolerant, using Apache ZooKeeper, is easy to scale and distribute the load.
\end{itemize}

Apache Solr is used as a base in many other systems, including some of the described in the following sections, and can also be used as a \ac{QA} system of its own\cite{ingersoll2013taming}.

\begin{figure}[!htbp]
    \centering
    \includegraphics[width=0.9\textwidth]{img/screens/solr-interface.png}
    \caption{Web interface for Solr queries.}
    \label{fig:solr1}
\end{figure}

% Add stanbol here?

\subsection{Linked Media Framework and Apache Marmotta}

Started as the Linked Media Framework\footnote{\url{https://bitbucket.org/srfgkmt/lmf/}}, this project aimed to provide an easy to setup server to offer linked media management, publishing linked data and allowing interactions with it. \ac{LMF} is built in modules, some of them optional, allowing the extension of the functionalities in the Linked Media Server. Some of the implemented modules are:

\begin{itemize}%[topsep=0pt,itemsep=-1ex,partopsep=1ex,parsep=1ex]
  \item \ac{LMF} Semantic Search allowing for a search service on top of Apache Solr. 
  \item \ac{LMF} Stanbol Integration, using Apache Stanbol for content analysis and interlinking.
  \item \ac{LMF} SKOS editor, to manage SKOS thesauruses imported in the Linked Media Server.
\end{itemize}

The core functionalities of \ac{LMF} where set aside to incubate Apache Marmotta, an Open Platform for Linked Data within the Apache Software Foundation~\footnote{\url{http://www.apache.org/}} aiming to provide an implementation for Linked Data that can be used to both publish Linked Data and build custom applications for Linked Data.

\subsection{Fuseki and Apache Jena}

Fuseki, build using Apache Jena\footnote{\url{http://jena.apache.org}}, is a \ac{SPARQL} server that provides a REST-style interface for \ac{SPARQL} queries. Its build using Apache Jena, an open source Semantic Web framework for Java. Apache Jena provides an API to extract data and write it to \ac{RDF} graphs, that are represented as an abstract model. A model can be sourced with data from files, databases, URLs or any combination of them. It also provides support for \ac{OWL}, and comes with several internal reasoners, as well as having support to use the Pellet\footnote{\url{https://www.w3.org/2001/sw/wiki/Pellet}} reasoner for \ac{OWL}

% Apache Marmotta / LMF
% DBpedia / virtuoso?
% Apache UIMA ?
% Apache Lucene / Solr / Stanbol ? 
% Terrier ? 



\section{Information retrieval}

The amount of information available in the web has grown exponentially over the last years, with standards such as Linked Data helping exchange data among heterogeneous systems. However, many times these standards are not used, and so it becomes necessary to recover and convert the data into compatible formats. There are multiple frameworks capable of crawling the web and recovering the relevant pieces of information, but we will focus here in Scrappy, a framework in ruby that allows extracting information from web pages and producing RDF data, and Scrapy, a python tool that allows extracting data from websites into any format, with powerful capabilities.

% Scrappy and scrapy

\subsection{Scrappy}
\label{subsec:scrappy}

Scrappy\cite{villamor13} is a ruby framework with multiple functionalities, including web and REST interface, storing the scrapped data in a RDF repository, and outputing in multiple formats. For our system, we will discuss the web scrapping and RDF output capabilities.

To recover data from a web page, Scrappy utilizes an ontology\footnote{http://www.gsi.dit.upm.es/ontologies/scraping/} to define the mapping between the web data and the semantic web resources. An example extractor can be found in listing \ref{listing:examplescrappy} will take the link to the GSI's staff page and return a foaf object with the name of each member. As seen in the extractor, it matches CSS elements in the page to semantic properties. Although not shown in this particular example, it can also use Xpath and regular expressions as selectors.

\begin{center} 
  \begin{lstlisting}[language={}, captionpos=b, caption=Example extractor, label=listing:examplescrappy]   
dc: http://purl.org/dc/elements/1.1/
rdf: http://www.w3.org/1999/02/22-rdf-syntax-ns#
sioc: http://rdfs.org/sioc/ns#
sc: http://www.gsi.dit.upm.es/ontologies/scraping.rdf#

_:gsipeople:
  rdf:type: sc:Fragment
  sc:type: foaf:Person
  sc:selector:
    *:
      rdf:type: sc:UriPatternSelector
      rdf:value: "http://www.gsi.dit.upm.es/index.php?option=com_jresearch&view=staff&layout=positions"
  sc:identifier:
    *:
      rdf:type: sc:BaseUriSelector
  sc:subfragment:
    *:
      sc:type: foaf:Person
      sc:selector:
        *:
          rdf:type: sc:CssSelector
          rdf:value: ".jrperson"
      sc:identifier:
        *:
          rdf:type: sc:CssSelector
          rdf:value: "a"
          sc:attribute: "href"
      sc:subfragment:
        *:
          sc:type: rdf:Literal
          sc:relation: foaf:givenName
          sc:selector:
            *:
              rdf:type: sc:CssSelector
              rdf:value: "a"
  \end{lstlisting}
\end{center}

An example of data generated with that extractor is shown in listing ~\ref{listing:examplerdf}. It can be seen how the elements matched in the extractor have been converted into properties. 

\begin{center}
  \lstinputlisting[language=XML, captionpos=b, caption=Example person extracted with Scrappy, label=listing:examplerdf]{code/enabling/person.rdf}
\end{center}

Using the appropriate selectors, scrappy can follow the links on a page, automating the scraping of big web sites, and converting all the data into RDF, N-Triples or JSON-LD.

\subsection{Scrapy}

Scrapy\footnote{\url{http://scrapy.org/}} is a fast high level web crawling framework, used to extract structured data from websites. It is written in python, and by default outputs data to json, although it accepts custom exporters, giving the user the ability to export into any format it requires.

The crawlers are python classes that extend the Spider class in scrapy.

\definecolor{Strings}{rgb}{0,0.63,0}
\begin{center}
  \lstinputlisting[language=Python, stringstyle=\color{Strings}, captionpos=b, caption=Example scrapy spider, label=listing:examplescrapy]{code/enabling/scrapy.py}
\end{center}

Listing \ref{listing:examplescrapy} shows part of a scrapy spider that will return a json object containing the url of the scrapped page, as well as the title, alternative and concept fields scrapped from the document.

\section{Web technologies}

% Talk about all-web movement? JavaScript & JQuery? 

% This is mostly bullshit. Rewrite it.

Known simply as ``The web'', the World Wide Web is an information system where hypertext documents are accessed via the internet. First proposed by Tim Berners Lee in 1989\cite{berners1989information}, it has grown to be used by two in five people around the world\footnote{\url{http://webfoundation.org/about/vision/history-of-the-web/}}.

The technologies used in web services can be divided in Client technologies, executed in the user's computer, and Server technologies, executed in the server side of the service. We will provide a short description of some of the technologies available for each side, focussing on those used in this project.

\subsection{Client technologies}

Web browsers are usually responsible for running the code of a website. Usually, that code consists on CSS, HTML and JavaScript files, that are interpreted by the browser to present the page, and respond to the user actions.

\begin{itemize}
 \item \textbf{HTML}: or \emph{HyperText Markup Language} is the standard markup language used to create web pages. It consists on a collection of pairs of tags that identify the different elements on a page, describing the structure of the page. It can also include images and other objects, allowing for complex user interaction.
\item \textbf{CSS}: for \emph{Cascading Style Sheets} is a style sheet language, used mostly to describe the look and formatting of documents written in a markup language such as HTML. It is designed to allow separation of content from document presentation, providing more flexibility and control of the presentation, while also improving accessibility.
 \item \textbf{JavaScript} is a programming language used to run scripts to interact with the user inside the browser window. Along with JavaScript, its most used library\footnote{\url{http://libscore.com/\#libs}} is jQuery, which is designed to simplify many of the usual tasks performed with JavaScript.
\end{itemize}

This technologies are often used with Ajax (\emph{Asynchronous Javascript XML}), a group of web development techniques used to create asynchronous web applications. Using Ajax, web applications can interact with the server in the background, therefore not interfering with the behaviour and graphical display of the rest of the page.

% JS/jQuery, HTML5, CSS, etc

\subsection{Server technologies}

% HTTP Servers. Apache?

The interactions presented by the web client are the processed in the server side, usually communicating using HTTP. There are multiple applications capable of handling this interaction, known as http servers. Apache, NginX or Microsoft Windows Server\textregistered~are some of the most popular servers\footnote{\url{http://news.netcraft.com/archives/2015/05/19/may-2015-web-server-survey.html}}. 

For our service, we have used Apache\cite{apacheabout}, an Open Source HTTP-Server. First launched in 1995, it has continued development to this date, with version 2.4.12 being released on January 2015~\footnote{\url{http://httpd.apache.org/}}. It is currently developed and maintained by an open community of developers under the Apache Software Foundation, and made available in a wide variety of operating systems, including GNU/Linux and Microsoft Windows\textregistered. It features a module-based system, allowing the core functionality to be expanded by compiled modules. Some of the most popular modules include:

\begin{itemize}%[topsep=0pt,itemsep=-1ex,partopsep=1ex,parsep=1ex]
 \item \textbf{mod\_php} Enabling the use php to execute server side code, this module can be found in many Apache installations, allowing the deployment of services like WordPress or Joomla.
 \item \textbf{mod\_auth\_basic} Handling basic user authentication, this module allows the server administrator to block sections of the server from being accessed by the general public.
 \item \textbf{mod\_proxy} Allows the use of Apache as a proxy, masking other services behind it.
\end{itemize}

Over the last few years, there has been an important trend in the use of evented web technologies, a vision of the traditional web APIs complemented with other APIs that produce events, and provide with a callback mechanism, make the Web more like a giant network. Node.js\footnote{\url{https://nodejs.org/}} is one of the most popular environments to build this kind of applications.

\subsubsection{WSGI Servers in python}

\ac{WSGI} is a specification for simple interfaces between web servers and web applications for the python programming language. First defined in PEP 333\cite{pep0333}, and updated in PEP 3333\cite{pep3333}, it has been adopted as a standard for python web application development.

There are multiple implementations and frameworks of WSGI for python, some of the most popular are:

\begin{itemize}
 \item \textbf{Bottle} is a simple lightweight WSGI framework, focusing on simplicity. It is distributed as a single file module, with no other dependencies than the python standard library. However, it has capabilities to handle routing, easy access to web data such as cookies and http headers, and includes a built-in server for development. It also has support for templates, both with a built-in engine, and using external modules such as mako or jinja2.
 \item \textbf{Django} is a full-fledged python web framework, offering fast and scalable services, with multiple built-in options, such as security, administration tools. It is slightly higher level than other frameworks, and emphasizes reusability of components and plugins, as well as rapid development. % complete?
 \item \textbf{Flask}, a micro web application framework, based on the jinja2 template engine, and the Werkzeug WSGI toolkit. It focuses on providing a simple interface, whilst still providing multiple features such as RESTfull request dispatching, cookies and request handling and unicode support. It also includes native support for unit testing, as well as a development server and debugger. It also supports extensions for extra functionalities.
\end{itemize}

In our application, we have chose Apache as the gateway server, using mod\_wsgi, and Flask for the application itself. 
%\emph{\textcolor{red}{Give ``example'' apache config?}}

\section{Summary}

In this section, we have discussed the technologies related to the system we are developing.

First, we took a look at \emph{conversational agent} systems. We saw that AIML and its implementations provide a robust technology, that is still evolving despite its age. It has been widely used, and still has an large community and volume of users, and is working on the new AIML 2.0. We also considered ChatScript characteristics, with better performance and new language for writing bots, with a simpler syntax and features not present in AIML 1.

We then considered \emph{Question Answering} systems and its different approaches, also taking a look at some systems that use each approach.

As for \emph{Linked Data Systems}, we studied the concept of the Semantic Web and Linked Data, to then take a look at multiple systems used for it: Apache Solr, not a linked data system on its own, but used in many of them, the Linked Media Framework, a full fledged linked data servers built around Apache Marmotta, and Fuseki, a SPARQL server with reasoning capabilities build with Apache Jena.

After considering the Data indexing systems, we studied two \emph{Information Retrieval} tools: Scrappy, a ruby web crawler capable of exporting RDF documents from the data extracted from the web, and Scrapy, a python framework for web spiders, with support for multiple functionalities as well as allowing export of the data in personalised formats.

Finally, we took a quick look at \emph{web technologies}, both for client-side applications, as well as server side systems that would allow complex interactions with the user. We focussed specially in the server side, studying different implementations of WSGI modules for python, and considering their characteristics.
 
\include{3-architecture}
\chapter{Case study: Java elearning platforms}
\label{chap:usecasejava}

\begin{chapterintro}

In this chapter, we will describe the prototype we developed for a Java Bot following the architecture described in the previous chapter. We will start with a short overview of the implemented modules for the system, to then take a detailed look to each one of them, describing how they work, and how they connect to each other.
 
\end{chapterintro}

\cleardoublepage

\section{Overview of the system}

For this system, we developed a prototype utilizing the architecture explained in chapter \ref{chap:architecture}. To do so, we deployed the following modules and subsystems:

\begin{enumerate}
 \item A Javascript client, that will connect to the system and act as a user interface.
 \item A Front-end controller, written in Python, that will handle the interaction between the different modules.
 \item A chatbot using ChatScript, to handle question analysis and social dialogue.
 \item An Apache Solr instance, where all the semantic data will be loaded.
\end{enumerate}

In parallel to all this we developed both a web scraper using scrapy to recover the relevant data, and an uploader to post the data to Solr.

%\emph{\textcolor{red}{Add here a description of the process}} %http://ieeexplore.ieee.org/stamp/stamp.jsp?tp=&arnumber=6595093

\section{Overall process}

In this section we will provide a short explanation on how each module of the system interact with each other. Figure \ref{fig:lifecycle} shows a schematic view of the process the system follows, first for the retrieval of the information and, when the system if fully deployed, facing a query from a user.

\begin{figure}[!htbp]
    \centering
    \includegraphics[width=\textwidth]{img/prot/lifecycle.png}
    \caption{Overall cycle for the system.}
    \label{fig:lifecycle}
\end{figure}

First, the data from the Vademecum need to be scrapped into \ac{RDF} documents, in an automatic process that can be supervised by a human administrator, but can also be fully automated. More details on this process can be found on section \ref{sec:javascrapping}. The generated \ac{RDF} data is then stored into our Knowledge Base, so it can be easily queried. The process of scrapping the data and indexing it can be performed periodically to include new data as it appears.

Once the system is deployed, users can input questions using the interface explained in section \ref{sec:chatclient}. Questions from the users will be packaged in a JSON request and then sent to the controller, that will process the query.

The controller, described in section \ref{sec:frontendcon}, will start the process sending the question to the Chatbot (section \ref{sec:chatbot}). The Chatbot will process the Natural Language input from the user, and classify it, attending on whether the question is simple social dialogue or requires a knowledge base lookup. In the former case, the Chatbot will simply generate a Natural Language response and send it to the controller, which will process it and sent it back to the chat client and the user. 

For the questions requiring Knowledge Base lookup, the Chatbot will identify the topic of the question and will send an Out of Band command to the controller, asking for the lookup. The controller will then form the query to retrieve the requested data, and perform the query. With the retrieved information, the controller will form a new out of band command, and send it back to the Chatbot. It will then process the data, and form a natural language response, sending it back to the controller for its final processing before being returned to the user in the chat client. The detailed information about the Knowledge Base can be found in section \ref{sec:solr}.

\section{Scrapping process}
\label{sec:javascrapping}
% \emph{\textcolor{red}{Add a section describing the scrapped documents, and the rdf data?}}

The data for this prototype comes from Professor Jose A. Mañas' Java Vademecum\footnote{\url{http://dit.upm.es/~pepe/libros/vademecum/}}, converted into \ac{RDF} and JSON format using scrapy to be uploaded to a Solr instance. In this process, we will study the structure of a Vademecum document and match it to the schema described in section \ref{subsec:solrschema}. An example of that conversion is shown in figure \ref{fig:vademecum1}.

\begin{figure}[!htbp]
    \centering
    \includegraphics[width=\textwidth]{img/screens/vademecum-ex.png}
    \caption{Example mapping for the vademecum.}
    \label{fig:vademecum1}
\end{figure}

The process for that document is automated using Python, XPath and regular expressions, using jinja2\footnote{\url{jinja.pocoo.org}} templates to generate the data. The scrapy spider will go through every entry in the Vademecum and create a list of items, which will in turn be exported in the required format. We have considered two formats for the export, JSON and \ac{RDF}. Both of them are based on a taxonomy using SKOS and Dublin Core. However, it is worth noting that we are using JSON rather than JSON-LD, so we also provide the \ac{RDF} as Semanic Data.

\subsection{RDF for Java}

The first part of the process is to define a taxonomy for the Java elements in the Vademecum. We have chosen to do this using SKOS\footnote{\url{http://www.w3.org/TR/skos-reference/}} and Dublin Core\footnote{\url{http://dublincore.org/documents/dcmi-terms/}}, both of them used for the representation of documents as Linked Data. We have therefore modelled a single Vademecum page with the fields shown in table \ref{tab:taxonomyfields}. Along with the objects representing the Vademecum documents, our scrapper will generate a SKOS ConceptScheme, with all the concepts identified in the Vademecum. 

\begin{center}
  \begin{table}
    \begin{tabular*}{0.7\textwidth}{@{\extracolsep{\fill}} | c | c | p{0.7\textwidth} |}
      \hhline{|-|-|-|}
      \textbf{Parameter} & \textbf{Source} & \textbf{Description} \\ \hhline{|=|=|=|}
      title & Dublin Core &  The name of the document, in Spanish. \\ \hhline{|-|-|-|}
      alternative & Dublin Core & The English equivalent for the title. \\ \hhline{|-|-|-|}
      definition & SKOS & A short description of the term.  \\ \hhline{|-|-|-|}
      description & Dublin Core & The full text of the document.  \\ \hhline{|-|-|-|}
      exmaple & SKOS & An example of the element described.  \\ \hhline{|-|-|-|}
      related & SKOS & A topic related to the one being described. \\ \hhline{|-|-|-|}
      inScheme & SKOS & The SKOS concept scheme this element is part of.  \\ \hhline{|-|-|-|}
      broader & SKOS & The parent SKOS concept for this element. \\ \hhline{|-|-|-|}
      \end{tabular*}
    \caption{Fields used representing a Java Object.}
    \label{tab:taxonomyfields}
  \end{table}
\end{center}

The data will be exported in RDF/XML format. An example of an object can be shown in listing \ref{listing:examplerdfvademecum}. It's worth noting that the URL for the described document is present as the rdf:about field.

\begin{center} 
  \begin{lstlisting}[language=XML, captionpos=b, caption=Example extractor, label=listing:examplerdfvademecum]   
    <skos:Concept rdf:about="http://www.dit.upm.es/~pepe/libros/vademecum/topics/26.html">
        <dcterms:title xml:lang="es">Asignacion</dcterms:title>
        <dcterms:alternative xml:lang="en">assignment</dcterms:alternative>
        <skos:definition>Se llaman sentencias de asignacion a las que cargan un nuevo valor en una variable</skos:definition>
        <dcterms:description>Se llaman sentencias de asignacion a las que cargan un nuevo valor en una variable: El tipo de la variable debe ser igual al de la expresion en tipos primitivos: asignable por promocion (ver \" Promocion\") asignable por reduccion (ver \" Reduccion\") en objetos: asignable por Upcasting (ver \"Casting\") asignable por Downcasting (ver \"Casting\")</dcterms:description>
        <skos:related>http://www.dit.upm.es/~pepe/libros/vademecum/topics/122.html</skos:related> <skos:related>http://www.dit.upm.es/~pepe/libros/vademecum/topics/242.html</skos:related> <skos:related>http://www.dit.upm.es/~pepe/libros/vademecum/topics/247.html</skos:related> <skos:related>http://www.dit.upm.es/~pepe/libros/vademecum/topics/288.html</skos:related> <skos:related>http://www.dit.upm.es/~pepe/libros/vademecum/topics/47.html</skos:related> <skos:related>http://www.dit.upm.es/~pepe/libros/vademecum/topics/86.html</skos:related> <skos:related>http://www.dit.upm.es/~pepe/libros/vademecum/topics/47.html</skos:related> <skos:related>http://www.dit.upm.es/~pepe/libros/vademecum/topics/3.html</skos:related> <skos:related>http://www.dit.upm.es/~pepe/libros/vademecum/topics/27.html</skos:related> <skos:related>http://www.dit.upm.es/~pepe/libros/vademecum/topics/28.html</skos:related> 
        <skos:example>variable = expresion ;</skos:example>
        <skos:broader>concepto</skos:broader>        
        <skos:inScheme>VademecumScheme</skos:inScheme>
    </skos:Concept>
 \end{lstlisting}
\end{center}


\subsection{JSON data}

Aside from the generated \ac{RDF}, the scrapper has also de ability to generate JSON. However, this format comes with some limitations, such as not including the concept scheme, and not keeping the source of the fields. As can be seen in listing \ref{listing:examplejsonvademecum}, a single JSON document will take the URL as identifier, and have similar fields than the \ac{RDF}. The correspondence between the \ac{RDF} and the JSON fields is shown in table \ref{tab:rdfjsonfields}


\begin{center}
  \begin{table}
    \begin{tabular*}{0.7\textwidth}{@{\extracolsep{\fill}} | c | c | p{0.5\textwidth} |}
      \hhline{|-|-|-|}
      \textbf{Json field} & \textbf{RDF equivalent} & \textbf{Description} \\ \hhline{|=|=|=|}
      title & dcterms:title &  The name of the document, in Spanish. \\ \hhline{|-|-|-|}
      alternative & dcterms:alternative & The English equivalent for the title. \\ \hhline{|-|-|-|}
      definition & skos:definition & A short description of the term.  \\ \hhline{|-|-|-|}
      description & dcterms:description & The full text of the document.  \\ \hhline{|-|-|-|}
      example & skos:example & An example of the element described.  \\ \hhline{|-|-|-|}
      links & skos:related & A topic related to the one being described. \\ \hhline{|-|-|-|}
      id & & An unique identifier for this document, matching the html document. \\ \hhline{|-|-|-|}
      concept & skos:broader & The corresponding concept in the concept scheme. \\ \hhline{|-|-|-|}
      \end{tabular*}
    \caption{Json fields and its equivalence.}
    \label{tab:rdfjsonfields}
  \end{table}
\end{center}

This format will be used to upload the data about the Vademecum documents to Solr.

\begin{center} 
  \begin{lstlisting}[language=JSON, captionpos=b, basicstyle=\tiny\ttfamily, caption=Example JSON document from the Vademecum, label=listing:examplejsonvademecum]   
 "http://www.dit.upm.es/~pepe/libros/vademecum/topics/26.html": {
        "definition": "Se llaman sentencias de asignacion a las que cargan un nuevo valor en una variable", 
        "concept": "concepto", 
        "description": "Se llaman sentencias de asignacion a las que cargan un nuevo valor en una variable: El tipo de la variable debe ser igual al de la expresion en tipos primitivos:  asignable por promocion (ver \"Promocion\")  asignable por reduccion (ver \"Reduccion\")  en objetos:  asignable por Upcasting (ver \"Casting\") asignable por Downcasting (ver \"Casting\")", 
        "links": [
            "http://www.dit.upm.es/~pepe/libros/vademecum/topics/122.html", 
            "http://www.dit.upm.es/~pepe/libros/vademecum/topics/242.html", 
            "http://www.dit.upm.es/~pepe/libros/vademecum/topics/247.html", 
            "http://www.dit.upm.es/~pepe/libros/vademecum/topics/288.html", 
            "http://www.dit.upm.es/~pepe/libros/vademecum/topics/47.html", 
            "http://www.dit.upm.es/~pepe/libros/vademecum/topics/86.html", 
            "http://www.dit.upm.es/~pepe/libros/vademecum/topics/47.html", 
            "http://www.dit.upm.es/~pepe/libros/vademecum/topics/3.html", 
            "http://www.dit.upm.es/~pepe/libros/vademecum/topics/27.html", 
            "http://www.dit.upm.es/~pepe/libros/vademecum/topics/28.html"
        ], 
        "example": "variable=expresion;"
        "title": "Asignacion", 
        "alternative": "assignment", 
        "id": "26"
    }, 

  \end{lstlisting}
\end{center}


\section{Chat client}
\label{sec:chatclient}

In our prototype, the interaction with the system is done via a web client that provides a chat box and an iframe where the content is located. When the user first opens the page, it's greeted by the bot, and provided with a   short explanation of how the client works.

\begin{figure}[!htbp]
    \centering
    \includegraphics[width=0.9\textwidth]{img/screens/ask-client.png}
    \caption{Web interface for the client.}
    \label{fig:chat1}
\end{figure}

The client is made using web technologies: HTML, CSS and Javascript, and uses Ajax to communicate with the server, sending the user questions and handling the response, waiting for the user to send a question and then making the request to the server, as shown in listing \ref{listing:formclient}.

%\emph{\textcolor{red}{Add example of JSON request? Add html code?}}

\begin{center}
  \lstinputlisting[language=JavaScript, caption=Ajax performing the request to the controller, label=listing:formclient, firstline=77, lastline=118]{code/prot/ask.js}
\end{center}

Upon the user submitting a question through the interface, the client will send a GET request to the controller, as described in section \ref{sec:frontendcon}, and will receive a JSON response, containing the answer and the page to be shown to the user, if existent. An example response to the question ``¿Qué es un for?'' is shown in listing \ref{listing:jsonchatresponse}.

\begin{center} 
  \begin{lstlisting}[language=json, caption=Example response for the chat client, label=listing:jsonchatresponse]
   {
     "answer": [
		"Esto es lo que s\u00e9 sobre for",
		"Si quieres, creo que bucles for degenerados tienen algo que ver con esto"
	       ],
     "definition": "Los bucles for se ejecutan un n\u00famero determinado de veces",
     "links": [
                "http://www.dit.upm.es/~pepe/libros/vademecum/topics/3.html",
                "http://www.dit.upm.es/~pepe/libros/vademecum/topics/139.html",
                "http://www.dit.upm.es/~pepe/libros/vademecum/topics/140.html",
                "http://www.dit.upm.es/~pepe/libros/vademecum/topics/141.html",
                "http://www.dit.upm.es/~pepe/libros/vademecum/topics/142.html",
                "http://www.dit.upm.es/~pepe/libros/vademecum/topics/143.html"
                ] ,
     "resource": "http://www.dit.upm.es/~pepe/libros/vademecum/topics/138.html"
   }  
  \end{lstlisting}
\end{center}

Figure \ref{fig:chat2} shows the web interface after a few interactions with the user.

\begin{figure}[!htbp]
    \centering
    \includegraphics[width=0.9\textwidth]{img/screens/ask-client-asked.png}
    \caption{Web interface after a question.}
    \label{fig:chat2}
\end{figure}

\section{Front end controller}
\label{sec:frontendcon}
The Front end controller is the main control module in our system. It handles the requests received from the client described in \ref{sec:chatclient}, and proceeds to triggers the required modules, as well as executing the Out of Band commands received from each module. This module is provided as a web service, and therefore we have chosen Flask~\cite{flask0101} and Apache's mod\_wsgi~\cite{modwsgi} to deploy it. In the following subsections we will describe how it works as well as its work-flow structure.

\subsection{Functional Model}
\label{subsec:functmodel}

The function of this module is returning the answer to the user, formed as JSON, by triggering the appropriate modules and reacting to their responses. To do so, it follows a process explained in the UML diagram shown in Figure \ref{fig:fe-model1}.

\begin{figure}[!htbp]
    \centering
    \includegraphics[width=0.8\textwidth]{img/prot/activityDiagram.png} 
    \caption{UML diagram of the process followed by the controller.}
    \label{fig:fe-model1}
\end{figure}

\begin{itemize}
 \item \textbf{Request parsing:} The client sends the query as JSON using a HTTP request to the front-end controller. This JSON is recovered and processed into a Python dictionary.
 \item \textbf{Send to ChatBot:} The user query is sent to the ChatBot so it's processes and a response, either \ac{NL} or Out of Band, is generated.
 \item \textbf{Split \ac{NL} response and Ouf of Band commands} The response in the previous step is split in \ac{NL} and Out of Band commands, to process each one appropriately.
 \item \textbf{Out of Band command processing:} Read the Out of Band commands and take the appropriate steps for each one of them.
 \item \textbf{Solr Lookup:} If a lookup in the Knowledge base is required, send the query to Solr.
 \item \textbf{Form Response:} Once there are no more Out of Band  commands, the controller forms the actual response in JSON and send it to the user.
\end{itemize}

% Add the workflow of the controller.

\subsection{Structural Model}

In this section we will describe the structure followed designing the front end controller. In figure \ref{fig:fe-methods1} we show the method structure of the controller.
\begin{figure}[!htbp]
    \centering
    \includegraphics[width=0.4\textwidth]{img/prot/controllerStructure.png}
    \caption{Front end controller structure}
    \label{fig:fe-methods1}
\end{figure}

Now we will proceed to describe each of the methods shown in the figure.

\begin{itemize}
  \item \textbf{rootURL()} will be triggered whenever the base URL for the controller is requested. It will act in the same way than the qa() method.
  \item \textbf{qa()} this function will recover the request parameters, shown in table \ref{tab:fe-qparams}, and start the process to understand the user query, in order to return the appropriate response, whose parameters are shown on table \ref{tab:fe-rparams}.
  \item \textbf{runQuestion()} in this function, we will simply run the question once through ChatScript, and then start the main command processing.
  \item \textbf{runCommands()} Given the \ac{NL} response from ChatScript, it will split the Out of Band commands and start processing each one of them, as well as their responses, adding commands to the queue as needed.
  \item \textbf{processSolr()} For the Solr command, this method will construct the solr query, and keep sending it to Solr increasing its fuzziness level, until an answer is found, or a maximum fuzziness is reached, in which case it will return the empty Solr response.
  \item \textbf{splitCommands()} Taking a \ac{NL} sentence as a parameter, this method will return both a list with all the Out of Band commands in the sentence, as well as the \ac{NL} part of the sentence, or an empty string if the sentence was made only of Out of Band commands. 
  \item \textbf{processGambit()} when a direct search does not return any answer, the system will perform a broader search in Solr. This is further explained in section \ref{sec:solr}.
  \item \textbf{sendChatScript()} This function will process the interaction with ChatScript, both sending the questions and handling the responses. For more details, see section \ref{sec:chatbot}.
  \item \textbf{sendSolrDefault()} Given a question and no other parameters or information, this function will send said question to Solr, so it will go through the default processing. This is mainly unused.
  \item \textbf{sendSolrEDisMax()} When a gambit is needed, send the question to Solr using an \ac{eDisMax} query, which will search in different fields, valuing each field using a given weight. 
  \item \textbf{sendSolr()} Used by the rest of the Solr related methods, this function will handle sending the payload given as parameter to Solr, and returning the JSON response.
\end{itemize}

\begin{center}
  \begin{table}
    \begin{tabular*}{0.7\textwidth}{@{\extracolsep{\fill}} | c | c | p{0.5\textwidth} |}
      \hhline{|-|-|-|}
      \textbf{Parameter} & \textbf{Name} & \textbf{Description} \\ \hhline{|=|=|=|}
      question & User's question & The question submitted by the user. \\ \hhline{|-|-|-|}
      bot & Bot & The bot the query will be send to. \\ \hhline{|-|-|-|}
      username & The user & A random identifier for the user communicating with the bot. \\ \hhline{|-|-|-|}
      \end{tabular*}
    \caption{Parameters in the query received by the front end controller.}
    \label{tab:fe-qparams}
  \end{table}
\end{center}

\begin{center}
  \begin{table}
    \begin{tabular*}{0.7\textwidth}{@{\extracolsep{\fill}} | c | c | p{0.5\textwidth} |}
      \hhline{|-|-|-|}
      \textbf{Parameter} & \textbf{Name} & \textbf{Description} \\ \hhline{|=|=|=|}
      Answer & The Bot's response & The \ac{NL} response generated by the bot. \\ \hhline{|-|-|-|}
      Resource & An URL & The URL of the relevant document where the information is, if existent. \\ \hhline{|-|-|-|}
      \end{tabular*}
    \caption{Parameters in the query sent back to the client.}
    \label{tab:fe-rparams}
  \end{table}
\end{center}

Finally, we will describe the Out of Band commands that the system will be able to process.

\begin{itemize}
 \item \textbf{sendSolr} this command is issued whenever a question matches the pattern specified to request information about a Java topic.
 \item \textbf{solrResponse} after a search is performed in Solr, this command is issued to indicate whether or not a response has been found, and returning said response.
 \item \textbf{solrLinks} a special Solr search, looking for related topics in Solr.
 \item \textbf{solrLinksResponse} as a response to the solrLinks command, returns a list with the related topics.
 \item \textbf{gambit} perform an \ac{eDisMax} search in Solr, using the full user question.
 \item \textbf{gambitResponse} returns the title of the most relevant document found in the \ac{eDisMax} search.
 \item \textbf{gambitUnknown} issued when no document with a high enough score is found in Solr.
 \item \textbf{resource} sets the URL to be displayed in the final response returned to the client.
 \item \textbf{label} sets the title of the found document as the label for the response.
\end{itemize}

The syntax of the commands is specified in table \ref{tab:oob-commands}.
\begin{center}
  \centering
  \begin{table}
    \begin{tabular*}{0.7\textwidth}{@{\extracolsep{\fill}} | c | c | p{0.35\textwidth} |}
      \hhline{|-|-|-|}
      \textbf{Command} & \textbf{Syntax} & \textbf{Description} \\ \hhline{|=|=|=|}
      sendSolr & sendSolr \textit{reqfield} \textit{doctitle} & Searchs in solr for the \textit{doctitle} and returns the \textit{reqfield} field.  \\ \hhline{|-|-|-|}
      \multirow{2}{*}{solrResponse} & solrResponse \textit{unknown} & The requested document was not found in Solr. \\ \cline{2-3}
				     & solrResponse \textit{reqfield} \textit{response} & Returns as \textit{response} the data of the field \textit{reqfield}. \\ \hhline{|-|-|-|}
      solrLinks & solrLinks \textit{linklist} & Asks for a search in Solr for the name of the topics given as an uri in the linklist \\ \hhline{|-|-|-|}
      solrLinksResponse & solrLinksResponse \textit{nameslist} & Sets the response for the solrLinks command, returning the first name of the links. \\ \hhline{|-|-|-|}
      gambit & gambit \textit{topic}& Asks for a \ac{eDisMax} search on Solr, passing the full question. \\ \hhline{|-|-|-|}
      gambitResponse & gambitResponse \textit{gambittopic} & After performing an \ac{eDisMax} search, returns \textit{gambittopic} as the suggested topic. \\ \hhline{|-|-|-|}
      gambitUnknown & gambitUnknown & After performing an \ac{eDisMax} search, indicates that no relevant document has been found. \\ \hhline{|-|-|-|}
      resource & resource \textit{URL} & Sets \textit{URL} as the resource to be displayed in the client \\ \hhline{|-|-|-|}
      label & label \textit{topic} & Sets \textit{topic} as the concept of the response \\ \hhline{|-|-|-|}
      \end{tabular*}
    \caption{Parameters in the query sent back to the client.}
    \label{tab:oob-commands}
  \end{table}
\end{center}

\section{Chatbot}
\label{sec:chatbot}

The Chatbot handles the processing of the natural language input from the user, and controls the conversation. To do so, it uses the chat engine ChatScript, described in section \ref{subsec:chatscript}. We will first describe the Chatbot rules, and how they process the user input, and then proceed to describe how to launch and communicate with the ChatScript server.

\subsection{The rules}

Chatscript responds to a series of rules, specified in its topic files. These rules will match the user input and produce an appropriate response, or output a rejoinder if no rule is matched. A more in depth description of how Chatscript rules work can be found in section \ref{subsec:chatscript}.

We have separated our topics across several files, each containing related topics, as well as the control script for the bot. We will start describing the control process, shown in listing \ref{listing:cs-control}

\begin{center}
  \lstinputlisting[language=C++, captionpos=b, caption=Control process for ChatScript, label=listing:cs-control, firstline=16, lastline=80]{code/prot/control.top}
\end{center}

In this process, we first check if we are currently in any conversation and have any pending rejoinders ready to match the output. Then proceed to check for rules looking for matches in the JAVA and EJEMPLOS topics. If no answer is provided, we will look for gambits and proper responses in the current topic, and then in the rest of the topics with keywords associated. In the case no match have been found yet, the user input will be test against the keywordless topic, looking for both responses and gambits, and, finally, if there is still no response, a generic answer will be provided, asking the user to modify its question.

As we have just mentioned, there are several topics, both with and without keyword, spread across several files, containing both said topics and the sets of concepts used in the bot. These files and the topics contained in them are:

\begin{itemize}
 \item \textbf{topic.top} This file has most of the concepts used in the other files, as well as three generic topics:
 \begin{itemize}
  \item \emph{TENCUESTA}\footnote{Since we had a concept defined as ``encuesta'', we needed to differentiate this topic, thus the starting ``T''} to answer questions regarding the poll that will be presented to the user.
  \item \emph{EXAMENES} containing generic responses to questions about the exams.
  \item \emph{INTRO} with information that will allow the bot to identify himself.
 \end{itemize}
 \item \textbf{java.top} This file contains the two main topics for answering the Java questions.
 \begin{itemize}
  \item \emph{JAVA} will answer the questions related to Java concepts, as well as produce gambits and question the user when the bot has the control over the conversation.
  \item \emph{EJEMPLOS} a slight variation from the previous topic, this will handle example requests from the user.
 \end{itemize}
 \item \textbf{javaconcepts.top} contains the list of all the Java concepts that the bots knows about. It is automatically generated from the data stored in Solr.
 \item \textbf{insults.top} with the topic of the same name (INSULTOS), responds to insults and bad words input by the user.
 \item \textbf{estado.top} a single topic file, responsible of responding when the user enters information about himself, or asks the bot about its status.
\end{itemize}

All these files need to be compiled into binary data to be used by the ChatScript server.
% Explain how chatscript works.

\subsection{The server}
\label{sec:csserver}

ChatScript provides two ways of interaction. For debugging purposes, it has a command line interface, and can also be deployed as a service listening in a TCP socket for user input. This server is what we have choose to use to interact with ChatScript. The full deployment instructions can be found in the appendix, and we will proceed to describe here the communication process.

As stated ChatScript listens on a TCP socket, waiting for requests containing three null separated strings, described in table \ref{tab:cs-reqparams}

\begin{center}
  \centering
  \begin{table}
  \begin{center}
    \begin{tabular*}{0.6\textwidth}{@{\extracolsep{\fill}} | c | p{0.5\textwidth} |}
      \hhline{|-|-|}
      \textbf{Field} & \textbf{Description} \\ \hhline{|=|=|}
      user & The string identifying the user performing the question.  \\ \hhline{|-|-|}
      bot & The bot that the question is directed to, in case there are several bot available in the server. \\ \hhline{|-|-|}
      question & The user question \\ \hhline{|-|-|}
      \end{tabular*}
    \caption{Fields for the request to ChatScript}
    \label{tab:cs-reqparams}
    \end{center}
  \end{table}
\end{center}

The server will then return in the same connection the response generated using the process previously explained, and close the connection. If a new interaction is needed, a new socket will be opened, following the same process.
% \subsubsection{Using Spanish dictionaries}

%\emph{\textcolor{red}{It doesn't look viable to have spanish dicts in a timely manner.}}

\section{Solr instance}
\label{sec:solr}

In this section we will describe how the Solr instance is set up, as well as the schema and the search procedure. We are using Apache Solr 4.10.2 as a document and search engine. The search queries will be send by the controller described in section \ref{sec:frontendcon}. The server will contain the data recovered from the Vademecum, structured in documents (one for each topic), and will allow us to perform the required searches. We will first describe the schema for the aforementioned documents, 

% Add the API rest for solr here?

% Add the relevant info from the sorlconfig here?

% Solr - add the relevant schema portions and how they work.

% Add somewhere how we are parsing the documents into json data.
\subsection{Data schema}
\label{subsec:solrschema}

The scrapped data is stored using a Solr core containing every relevant document. This is done by describing the structure of said documents in the schema.xml for Solr. In this file, we can consider three main groups of fields:

\begin{itemize}
 \item \textbf{Stock solr fields: } These fields are internal for Solr, and we will not describe them here.
 \item \textbf{Document fields: } the fields scrapped from the Vademecum, they are described in table \ref{tab:schema-docfields}.
 \item \textbf{titleDefinition field: } this field is generated concatenating the definition and title fields, to facilitate general search. %\emph{\textcolor{red}{Describe the copy process?}}
\end{itemize}

The XML for these fields is as in listing \ref{listing:solr-schema1}
\begin{center}
  \lstinputlisting[language=XML, captionpos=b, caption=fields defined for the Java documents in the schema, label=listing:solr-schema1, firstline=125, lastline=148]{code/prot/schema.xml}
\end{center}

\begin{center}
  \centering
  \begin{table}
  \begin{center}
    \begin{tabular*}{0.655\textwidth}{@{\extracolsep{\fill}} | c | c | p{0.3\textwidth} |}
      \hhline{|-|-|-|}
      \textbf{Field} & \textbf{Field Type} & \textbf{Description} \\ \hhline{|=|=|=|}
      title & text\_search\_es & The title of this particular document.  \\ \hhline{|-|-|-|}
      alternative & lowercase & If exists, the English name for the document. \\ \hhline{|-|-|-|}
      concept & lowercase & What concept does this document refers to. \\ \hhline{|-|-|-|}
      resource & string & The link for this document. \\ \hhline{|-|-|-|}
      definition & text\_search\_es & The first sentence of the document. \\ \hhline{|-|-|-|}
      description & text\_search\_es & The text of the document. \\ \hhline{|-|-|-|}
      links & text\_ws & Related documents to this one. \\ \hhline{|-|-|-|}
      examples & text\_general & The scrapped text of the examples. \\ \hhline{|-|-|-|}
      \end{tabular*}
    \caption{Fields for the documents stored in the Solr schema.}
    \label{tab:schema-docfields}
    \end{center}
  \end{table}
\end{center}

As shown in the table, the fields have a ``field type'' associated, that will determine how they are stored and how they are tokenized and processed when performing the indexing and the search. The fieldTypes we are using are as follow:

\begin{itemize}
 \item \textbf{ string } This field is a default Solr field, and stored verbatim using the solr.StrField class.
 \item \textbf{ text\_general } A default Solr field, using the sorl.TextField class.
 \item \textbf{ lowercase } A variation on the default lowercase fieldType by Solr, it used the default keyword tokenizer and the lowercase filter, and we have added the ASCIIFolding filter.
 \item \textbf{ text\_search\_es } Based on Solr's Spanish fields, this field will contain general text to perform a search. It uses a standard tokenizer, as well as the following filters:
 \begin{itemize}
  \item Lowercase filter: converts all words to lowercase 
  \item Stopfilter factory using Spanish stopwords and the snoball stemming algorithm
  \item Spanish light stem filter, a default solr stemmer for Spanish.
  \item Only for the query processing, a worddelimiter filter to do the \ac{QA}.
 \end{itemize}
\end{itemize}

The listing \ref{listing:solr-schema2} shows the description of the text\_search\_es fieldType.

\begin{center}
  \lstinputlisting[language=XML, captionpos=b, caption=Definition for the text\_search\_es fieldType, label=listing:solr-schema2, firstline=542, lastline=567]{code/prot/schema.xml}
\end{center}

With this configuration we will be able to do the queries described in the following sections.

% Add something about the solr config?

\subsection{Faceted query}
% Maybe change the title?

In the event that ChatScript identifies the question and the topic the user is asking for, we will perform a faceted search in Solr, looking for the fields requested in the Out of Band command. The query will be done in JSON format. An example of a query is shown in listing \ref{listing:solrquery1}, and the meaning of each field is described next:

\begin{center} 
  \begin{lstlisting}[language=json, caption=Example JSON query for Solr, label=listing:solrquery1]
   {
     "q" : "title:for~0",
     "wt" : "json",
     "fl" : "*,score",
     "rows" : "1"
   }  
  \end{lstlisting}
\end{center}

\begin{itemize}
 \item \textbf{q} contains the actual query sent to the server, with the filter for the query and the fuzziness. In the example, we are searching for documents containing the given string in the title, with a fuzziness of 0 (an exact match).
 \item \textbf{wt} the format the data will be returned in. In the example, we want the data in JSON format.
 \item \textbf{fl} the list of fields we want for the documents in the response. In the example, this is set to all the fields in the document, specified in the query as a wild card ``*'', as well as the score for the match.
 \item \textbf{rows} the number of documents to return.
\end{itemize}

With this search, we will try to look up the documents when the ChatScript module clearly identifies the topic the user is asking about, and, properly defining the returning fields for the search, the controller will show the relevant data. In case the question is not clearly identified, but ChatScript recognizes the user is talking about some topic related to Java, this query won't be valid, and we will have to perform an \ac{eDisMax} query, described in the next section.

\subsection{Gambit query}
\label{subsec:solrgambit}
% Explain the process to run the eDisMax query

When ChatScript identifies a sentence talking about a Java concept, but does not recognize a question, we will perform a search in Solr using the entire question, treating it as a \ac{QA} system. To do so, we will perform an \ac{eDisMax} query, taking advantage of the stemmers and tokenizers we have configured in section \ref{subsec:solrschema}. This type of query is designed to process user input directly, searching for the keyword across multiple fields, with different boosts based on the significance of each field, and allowing multiple options to influence the score on a case to case basis.

Like the regular Solr query, this query is perform using a JSON format, and sent to the Solr service as the payload to a GET request. An example of an \ac{eDisMax} query is shown in listing \ref{listing:solrqueryeDisMax}

\begin{center} 
  \begin{lstlisting}[language=json, caption=Example \ac{eDisMax} query for Solr, label=listing:solrqueryeDisMax]
   {
     "q": "hablame de los bucles for",
     "defType": "edismax",
     "qf": "title^10.0 description^2.0",
     "fl": "*,score",
     "rows": "1",
     "wt": "json",
     "lowercaseOperators": "true",
     "stopwords": "true"
   }  
  \end{lstlisting}
\end{center}

The significance of each in the query is as follows:

\begin{itemize}
 \item \textbf{q} The question as sent by the user, with no processing.
 \item \textbf{defType} Explicitly set the query type as \ac{eDisMax}.
 \item \textbf{qf} Set the weights for each field to consider in the query.
 \item \textbf{fl} The fields to return.
 \item \textbf{rows} The number of documents to return.
 \item \textbf{wt} The format for the response.
 \item \textbf{lowercaseOperators} Interpret lowercase words as boolean operators, such as ``and'' and ``or''.
 \item \textbf{stopwords} Use the stopwords defined in the schema.
\end{itemize}

This search will provide a broader match than the faceted query, finding matches when the topic of the question is not clear, and offering the answer to the user. To prevent completely unrelated topics to be offered to the user, the score is retrieved and only the answers with a minimum score will be returned.

\section{Summary}

In this Chapter, we have described a prototype of the proposed architecture for a system focussing on supporting lessons about the Java Programming Language.

We started providing a general overview of the system, describing how each module is related to the others.

We then talked about the \emph{information retrieval} module, and how the information about the Java programming language was recovered from the Java Vademecum, using scrapy, and how we mapped it to a \ac{RDF} taxonomy. We then studied the JSON format that would be used to upload the data to Solr.

We moved onto the \emph{web interface} that presents the system to the final user, explaining the interface and how the system works to the final user. We also presented the interactions of the client with the server. 

The client connects with the \emph{Front end controller}, and we studied its implementation. We showed how the controller connects with the other modules, as well as analysed the Out of Band commands the different modules use to communicate with each other, and the workflow of the controller.

Taking a look at the \emph{conversational agent}, we studied ChatScript and the rules used in our system, as well as how they were triggered and the output they produce.

Finally, we analysed the \emph{Solr instance} configuration, taking a close look at the schema and the fields for the stored documents. We then proceeded to discuss the types of queries the system, the regular faceted query for user questions, and the eDisMax queries for gambits.
\chapter{Case study: GSI Bot}
\label{chap:usecasegsi}

\begin{chapterintro}

In this chapter, we will describe the prototype we developed for a bot for the GSI web page following the architecture described in chapter \ref{chap:architecture}. We will start with the process to recover the data as Linked Data, to them describe the interface and the modules of the system.
 
\end{chapterintro}

\cleardoublepage

\section{Overview of the system}

Along with the prototype described in chapter \ref{chap:usecasejava}, we have also developed a system with the data from the GSI webpage, including data from the projects, publications and staff. For this, we have similar modules:

\begin{enumerate}
 \item A Javascript client acting as the user interface.
 \item A Python controller, handling the flow of the information in the system
 \item A different ChatScript bot, handling the conversation.
 \item An Apache Solr core, with all the data.
\end{enumerate}

In this case, the data was recovered using a mix of techniques, and integrated into a single core in Solr.

\section{Recovering and storing the data}

Similarly to the previous chapter, the data for this prototype has been recovered and converted into RDF and json formats, 

\subsection{Webpage categories}

We considered three types of data from the GSI web page: the information about the members of the groups, their publications and the projects the groups has taken part of. Each type comes from a different part of the website, and therefore will be considered independently. 

\subsubsection{Projects}

For the projects information, the data is available in different formats in the webpage itself, including RDF/XML, as shown in figure \ref{fig:gsiprojects}

\begin{figure}[!htbp]
    \centering
    \includegraphics[width=0.8\textwidth]{img/screens/gsi-projects.png}
    \caption{RDF exporter for the projects.}
    \label{fig:gsiprojects}
\end{figure}


\subsubsection{Publications}
\label{subsec:bibtexrdf}

Each publication listed in the GSI webpage has an attached bibtex citation. Therefore, it is possible to use a bibtex ontology\footnote{\url{http://purl.oclc.org/NET/nknouf/ns/bibtex}} to map the elements in the bibtex files to semantic data. The mapping for the classes is shown in table \ref{tab:bibtex-classes}.

\begin{center}
  \begin{table}
    \begin{tabular*}{0.7\textwidth}{@{\extracolsep{\fill}} | c | c | p{0.5\textwidth} |}
      \hhline{|-|-|-|}
      \textbf{Bibtex tag} & \textbf{Ontolgoy mapping} & \textbf{Description} \\ \hhline{|=|=|=|}
      article & bibtex:Article & An article from a journal or magazine. \\ \hhline{|-|-|-|}
      book & bibtex:Book & A book with an explicit publisher. \\ \hhline{|-|-|-|}
      conference & bibtex:Conference & An article in a conference proceedings. \\ \hhline{|-|-|-|}
      inbook & bibtex:Inbook & A part of a book, which may be a chapter (or section or whatever) and/or a range of pages. \\ \hhline{|-|-|-|}
      incollection & bibtex:Incollection & A part of a book having its own title. \\ \hhline{|-|-|-|}
      masterthesis & bibtex:Masterthesis & A Master's thesis \\ \hhline{|-|-|-|}
      phdthesis & bibtex:Phdthesis & A PhD thesis. \\ \hhline{|-|-|-|}
      proceedings & bibtex:Proceedings & The proceedings of a conference \\ \hhline{|-|-|-|}
      techreport & bibtex:Techreport & A report published by a school or other institution, usually numbered within a series. \\ \hhline{|-|-|-|}
      \end{tabular*}
    \caption{Classes for the bibtex documents.}
    \label{tab:bibtex-classes}
  \end{table}
\end{center}

As seen in listing \ref{listing:examplerdfbibtex}, the mapping for the properties follows a simple pattern, similar to the clases' mapping. To this mapping we have added the source of the document using Dublin Core, so the original bibtex can be referenced as needed.

\begin{center} 
  \begin{lstlisting}[language=XML, captionpos=b, caption=Example bibtex document converted to RDF, label=listing:examplerdfbibtex]   
   <bibtex:Conference rdf:about="gsi:serranoTwitter2015">
       <bibtex:hasTitle>A survey of Twitter Rumor Spreading Simulations</bibtex:hasTitle>
       <bibtex:hasAuthor>gsi:eserrano</bibtex:hasAuthor>
       <bibtex:hasAuthor>gsi:ciglesias</bibtex:hasAuthor>
       <bibtex:hasAuthor>gsi:mgarijo</bibtex:hasAuthor>
       <dcterms:source>http://www.gsi.dit.upm.es/index.php/es/investigacion/publicaciones.bibtex?controller=publications&amp;task=export&amp;id=364</dcterms:source>
       <bibtex:hasYear>2015</bibtex:hasYear>
       <bibtex:hasMonth>September</bibtex:hasMonth>
       <bibtex:hasBooktitle>7th International Conference on Computational Collective Intelligence Technologies and Applications</bibtex:hasBooktitle>
    </bibtex:Conference>
 \end{lstlisting}
\end{center}

\subsubsection{People}

Finally, the information available in the GSI web page about the members of the group cannot be exported in any format, so we have used scrappy, described in section \ref{subsec:scrappy} to crawl the data and export it into an appropriate format. In order to do so, we used the \ac{FOAF}\footnote{\url{http://xmlns.com/foaf/spec/}} ontology as well as some elements from the \ac{SIOC} ontology\footnote{\url{http://rdfs.org/sioc/spec/}} to represent each person in the group. An example mapping can be found in listing \ref{listing:examplerdfperson}

\begin{center} 
  \begin{lstlisting}[language=XML, captionpos=b, caption=Example semantic data about a member of the group, label=listing:examplerdfperson]   
  <foaf:Person rdf:about="gsi:amardomingo">
    <foaf:workInfoHomepage>http://www.gsi.dit.upm.es/index.php?option=com_jresearch&amp;view=member&amp;task=show&amp;id=84</foaf:workInfoHomepage>
    <sioc:Role>5&gt; Becario de Grado</sioc:Role>
    <foaf:givenName>Alberto Mardomingo Mardomingo</foaf:givenName>
    <foaf:homepage>http://gsi.dit.upm.es/~amardomingo/</foaf:homepage>
    <foaf:img>http://www.gsi.dit.upm.es/uploads/jresearch/assets/members/Foto.jpg</foaf:img>
  </foaf:Person>
 \end{lstlisting}
\end{center}

This data was generated using scrappy, running the extractor shown in listing \ref{listing:peoplescrappy}.

\begin{center} 
  \begin{lstlisting}[language={}, captionpos=b, caption=Extractor for the GSI people section, label=listing:peoplescrappy]   
_:gsipeople:
  rdf:type: sc:Fragment
  sc:type: foaf:Person
  sc:selector:
    *:
      rdf:type: sc:UriPatternSelector
      rdf:value: "http://www.gsi.dit.upm.es/index.php?option=com_jresearch&view=staff&layout=positions"
  sc:identifier:
    *:
      rdf:type: sc:BaseUriSelector
  sc:subfragment:
    *:
      sc:type: foaf:Person
      sc:selector:
        *:
          rdf:type: sc:CssSelector
          rdf:value: ".jrperson"
      sc:identifier:
        *:
          rdf:type: sc:CssSelector
          rdf:value: "a"
          sc:attribute: "href"
      sc:subfragment:
        *:
          sc:type: rdf:Literal
          sc:relation: foaf:givenName
          sc:selector:
            *:
              rdf:type: sc:CssSelector
              rdf:value: "a"
_:gsiperson:
  rdf:type: sc:Fragment
  sc:type: foaf:Person
  sc:selector:
    *:
      rdf:type: sc:UriPatternSelector
      rdf:value: "http://www.gsi.dit.upm.es/index.php?option=com_jresearch&view=member&task=show&id=*" 
  sc:identifier:
    *:
      rdf:type: sc:BaseUriSelector
  sc:subfragment:
    *:
      sc:type: rdf:Literal
      sc:relation: foaf:homepage
      sc:selector:
        *:
          rdf:type: sc:CssSelector
          rdf:value: ".personalpage a"
    *:
      sc:type: rdf:Literal
      sc:relation: foaf:img
      sc:selector:
        *:
          rdf:type: sc:CssSelector
          rdf:value: ".persimg"
          sc:attribute: "href"
    *:
      sc:type: rdf:Literal
      sc:relation: foaf:phone
      sc:selector:
        *:
          rdf:type: sc:CssSelector
          rdf:value: ".jrpf"
     *:
      sc:type: rdf:Literal
      sc:relation: sioc:Role
      sc:selector:
        *:
          rdf:type: sc:CssSelector
          rdf:value: ".jrposition"
     *:
      sc:type: rdf:Literal
      sc:relation: foaf:based_near
      sc:selector:
        *:
          rdf:type: sc:CssSelector
          rdf:value: ".jrlocation"

  \end{lstlisting}
\end{center}

\section{User interface}

For this system, the user interface is similar to the one described in section \ref{sec:chatclient}, with some minor changes. 

\begin{figure}[!htbp]
    \centering
    \includegraphics[width=0.7\textwidth]{img/screens/gsi-client.png}
    \caption{Web interface for the client.}
    \label{fig:gsichat1}
\end{figure}

\section{Controller}
\label{sec:controllergsi}

The controller handles the requests received from the client described in the previous section, interacting with the required modules as needed, and interpreting the \ac{OoB} commands generated by each module. This module is very similar to the one described in section \ref{sec:frontendcon} of the previous chapter, and is developed using the same technologies. It follows the same functional model described in section \ref{subsec:functmodel}.

\subsection{Structural Model}

In this section we will describe the structure followed by the controller of this prototype. The relevant methods of said structure are shown in figure \ref{fig:gsi-methods1}.

\begin{figure}[!htbp]
    \centering
    \includegraphics[width=0.4\textwidth]{img/prot2/controllerStructure.png}
    \caption{Front end controller structure}
    \label{fig:gsi-methods1}
\end{figure}

The function for each of those methods is described next:

\begin{itemize}
  \item \textbf{rootURL()} triggered when the base URL for the controller is requested, it acts in the same way as the qa() method.
  \item \textbf{qa()} Parsing the request parameters shown in table \ref{tab:fe-qparams}, this method will start the process to process the user question, and return the response once said process is complete, following the structure on table on table \ref{tab:fe-rparams}. 
  \item \textbf{runQuestion()} Send the question to ChatScript once, to generate the first \ac{OoB} commands and start processing them.
  \item \textbf{runCommands()} Parse the response from ChatScript and split the \ac{OoB} commands, in order to start processing each one of them, as well as their responses, adding commands to the queue as needed.
  \item \textbf{solrPublication()} When the user is asking about a publication, parse the parameters and perform the relevant Solr lookup.
  \item \textbf{solrProject()} Parse handle the queries to Solr when the question involves projects.
  \item \textbf{solrPerson()} Handle requests regarding members of the GSI.
  \item \textbf{solrCount()} When the user ask about quantities rather than about the documents themselves, perform the appropiate Solr lookup.
  \item \textbf{processGambit()} As a last resource, perform a broad search in Solr with the user query. This is further explained in section \ref{sec:solr}.
  \item \textbf{sendChatScript()} This function will process the interaction with ChatScript, both sending the questions and handling the responses. For more details, see section \ref{sec:chatbot}.
  \item \textbf{sendSolrDefault()} Given a question and no other parameters or information, this function will send said question to Solr, so it will go through the default processing. This is mainly unused.
  \item \textbf{sendSolrEDisMax()} Send the question to Solr using an \ac{eDisMax} query, usually for a gambit. 
  \item \textbf{sendSolr()} Send a direct query to solr. Mainly used by the rest of the solr lookup methods.
\end{itemize}

Finally, we will describe the \ac{OoB} commands that the system will be able to process.

\begin{itemize}
 \item \textbf{¬solrPublication} and \textbf{¬solrResponsePublication} this commands handle the interactions when the user question is about the publications.
 \item \textbf{¬solrProject} and \textbf{¬solrResponseProjects} are issued for the query and response when performing lookups about the projects.
 \item \textbf{¬solrPerson} and \textbf{¬solrResponsePerson} handle the queries for the lookups related with the GSI members.
 \item \textbf{¬solrCount} and \textbf{¬solrcounted} perform the query for the number of documents relevant to a given filter.
 \item \textbf{¬solrLinks} a special Solr search, looking for related topics in Solr. 
 \item \textbf{¬gambit} performs an \ac{eDisMax} search in Solr, with the user query.
 \item \textbf{¬gambitResponse} returns the response of the gambit \ac{eDisMax} query.
 \item \textbf{¬gambitUnknown} issued when the gambit does not return any relevant document.
 \item \textbf{¬resource} sets the URL to be shown to the user in the client.
\end{itemize}

The syntaxes of the commands is specified in table \ref{tab:gsi-commands}. \emph{\textcolor{red}{Update this table}}
\begin{center}
  \centering
  \begin{table}
    \begin{tabular*}{0.7\textwidth}{@{\extracolsep{\fill}} | c | c | p{0.35\textwidth} |}
      \hhline{|-|-|-|}
      \textbf{Command} & \textbf{Syntaxes} & \textbf{Description} \\ \hhline{|=|=|=|}
      ¬sendSolr & ¬sendSolr \textit{reqfield} \textit{doctitle} & Searchs in solr for the \textit{doctitle} and returns the \textit{reqfield} field.  \\ \hhline{|-|-|-|}
      \multirow{2}{*}{¬solrResponse} & ¬solrResponse \textit{unknown} & The requested document was not found in Solr. \\ \cline{2-3}
				     & ¬solrResponse \textit{reqfield} \textit{response} & Returns as \textit{response} the data of the field \textit{reqfield}. \\ \hhline{|-|-|-|}
      ¬solrLinks & ¬solrLinks \textit{linklist} & Asks for a search in Solr for the name of the topics given as an uri in the linklist \\ \hhline{|-|-|-|}
      ¬solrLinksResponse & ¬solrLinksResponse \textit{nameslist} & Sets the response for the ¬solrLinks command, returning the first name of the links. \\ \hhline{|-|-|-|}
      ¬gambit & ¬gambit \textit{topic}& Asks for a \ac{eDisMax} search on Solr, passing the full question. \\ \hhline{|-|-|-|}
      ¬gambitResponse & ¬gambitResponse \textit{gambittopic} & After performing an \ac{eDisMax} search, returns \textit{gambittopic} as the suggested topic. \\ \hhline{|-|-|-|}
      ¬gambitUnknown & ¬gambitUnknown & After performing an \ac{eDisMax} search, indicates that no relevant document has been found. \\ \hhline{|-|-|-|}
      ¬resource & ¬resource \textit{URL} & Sets \textit{URL} as the resource to be displayed in the client \\ \hhline{|-|-|-|}
      ¬label & ¬label \textit{topic} & Sets \textit{topic} as the concept of the response \\ \hhline{|-|-|-|}
      \end{tabular*}
    \caption{Parameters in the query sent back to the client.}
    \label{tab:gsi-commands}
  \end{table}
\end{center}


\section{Chatbot}
\label{sec:chatbotgsi}

Using the ChatScript chat engine, the Chatbot handles the conversation and process the natural language interactions with the user. However, to adapt to the fact that the documents stored in Solr for this prototype are not homogeneous, the rule structure has been changed.

\subsection{The rules}

We have separated the rules regarding different types of documents in one topic file per type, and therefore, we will have the following files and topics:

We have separated our topics across several files, each containing related topics, as well as the control script for the bot. We will start describing the control process, shown in listing~\ref{listing:cs-control}


\begin{itemize}
 \item \textbf{people.top} This file has the interactions regarding the members of the group, in the PEOPLE topic.
 \item \textbf{projects.top} with the topics regarding the projects.
 \item \textbf{publications.top} contains the topics relevant to the publications.
 \item \textbf{introductions.top} contains simple chat about the bot, as well as greetings and farewells.
 \item \textbf{mixed.top} in this file are stored topics with questions regarding more than one of the fields.
\end{itemize}

Finally, the control script used for this bot is very similar to the one provided by default with ChatScript, since we do not have the limitations regarding changing the language in this bot.

This control script will first look for rejoinders in the current topic, to then proceed to offer a gambit if no rejoinder is found and the keyword match. If there is no gambit available for the current interaction, the system will look for standard responders, both in this topic and in any other topic whose keywords match the current volley. If no topic reacts, the systems will then try to match the responders from the keywordless topics, to finally gambit any keyword topic. If there is no response at that point, it forces a gambit from the current topic, to finally offer a generic response, indicating the system does not know how to respond.


\subsection{The server}

The ChatScript server for this prototype is deployed in the same way than the server in the previous chapter, explained in section \ref{sec:csserver}. However, it is worth mentioning that a single ChatScript instance cannot hold both bots, so if the same physical server is to run both bots, it would be needed to run two ChatScript instances, each one of them in a different port.

\section{Solr instance}

In this section we will describe how the Solr instance has been configured for this system, as well as the search procedure. As in section \ref{sec:solr}, we are using Apache Solr 4.10.2, which will receive the queries from the controller described in section \ref{sec:controllergsi}. The Solr instance will contain a core with the data scrapped from the web page, allowing us to have all the date in a single place.

The data was merged into a single RDF file, to then be imported and uploaded into the Solr instance.

\subsection{Solr Schema}

For this system, we have stored all the scrapped data in the same Solr core. Therefore, we can classify the fields according to the type of document they are associated with. Therefore, we will look at the fields grouping them according to said type of document:

\begin{itemize}
 \item \textbf{Stock solr fields:} Fields internal for Solr, we will not describe them here.
 \item \textbf{Document fields:} the fields scrapped from the web.
    \subitem \emph{Fields for the projects:} These fields are derived from the project exported structure.
    \subitem \emph{Fields for the members of the GSI}: Describing the people at the GSI, this fields are based on \ac{FOAF} fields.
    \subitem \emph{Fields for the publications:} Based on the bibtex ontology described in section \ref{subsec:bibtexrdf}.
 \item \textbf{searchField field:} this fields is generated concatenating fields from all three types of fields, to allow a default search with a single field.
\end{itemize}

A short description of each field, as well as the field types is shown in table \ref{tab:schema-gsifields}

\emph{\textcolor{red}{Fill this table. Maybe move it to an appendix?}}
\begin{center}
  \centering
  \begin{table}
  \begin{center}
    \begin{tabular*}{0.8\textwidth}{@{\extracolsep{\fill}} | c | c | c | p{0.3\textwidth} |}
      \hhline{|-|-|-|-|}
      \textbf{Field} & \textbf{Field Type} & \textbf{Type of document} & \textbf{Description} \\ \hhline{|=|=|=|=|}
      about & string & all & The object this document is describing. \\ \hhline{|-|-|-|-|}
      type & string & all & The object class for this document. \\ \hhline{|-|-|-|-|}
      journal & lowercase & Publication & The Journal associated with the publication. \\ \hhline{|-|-|-|-|}
      volume & string & Publication & The volume number \\ \hhline{|-|-|-|-|}
      year & string & Publication & The year for the publication \\ \hhline{|-|-|-|-|}
      month & lowercase & Publication & The month for the publication \\ \hhline{|-|-|-|-|}
      booktitle & text\_search & Publication & The title of the book the publication is associated with. \\ \hhline{|-|-|-|-|}
      --- & --- & --- & -- \\ \hhline{|-|-|-|-|}
      \end{tabular*}
    \caption{Fields in the Solr schema for the gsidata core.}
    \label{tab:schema-gsifields}
    \end{center}
  \end{table}
\end{center}

The ``text\_search'' field is based on the default English fields, and is defined as shown in listing \ref{listing:schemagsisearch}

\begin{center}
  \lstinputlisting[language=XML, captionpos=b, caption=Definition for the text\_search fieldType, label=listing:schemagsisearch, firstline=580, lastline=599]{code/prot2/schema.xml}
\end{center}

With this configuration we will perform the queries described in the next sections.

\subsection{Solr queries}

For this system, we have considered several types of queries. We have considered questions about the different objects and its relations, questions about quantifying specific subsets, and gambit queries. 

\subsubsection{Questions about quantities}

This query is perform when ChatScript identifies a question about the about of some type of document. This query will recover the number of documents in Solr matching the required criteria, and return it in an \ac{OoB} command. For example, for a question about the number of publications in 2014, the query will be as shown in listing \ref{listing:solrgsi1}. The meaning of each field is described next:

\begin{center} 
  \begin{lstlisting}[language=json, caption=Example json query for Solr, label=listing:solrgsi1]
   {
     "q" : "type:*bibtex* AND year:2014",
     "wt" : "json",
     "rows" : "0"
   }  
  \end{lstlisting}
\end{center}

\begin{itemize}
  \item \textbf{q} contains the actual query sent to the server, specifying the fields we are filtering with. In the example, we are searching for documents for year 2014. Since ``year'' is a field unique to publications, there is no real need to add the type filter.
  \item \textbf{wt} the format the data will be returned in. In the example, we want the data in json format.
  \item \textbf{rows} the number of documents to return. Since we only want the actual result count, rather than the documents, in the example it is set to 0
\end{itemize}

This search will return a json with the number of documents matching the criteria, which in turn will be pass to ChatScript, generating the appropriate \ac{NL} response.

\subsubsection{General questions}

ChatScript can also identify general questions about the different objects stored in Solr. In those cases, Solr will perform a regular search, returning the document with the highest score, which will them offered to the user as a response.

\emph{\textcolor{red}{Add example query for this}}

\subsubsection{Gambit queries}

In the event that ChatScript is incapable of identify the question the user is making, the system will perform an \ac{eDisMax} query, looking for a match in the relevant fields for each document type, and offering the answer only if the score of the match is over a predetermined minimum score.

\section{Summary}

In this chapter we have talked about a different prototype for our system, including documents with different structures, indexed into the system.

We first studied the three main sources of data, proposing a RDF mapping to export the scrapped data into semantic data. We also discuss the different techniques and tools used while recovering the data. We recovered the data from the GSI members, their publications and their projects.

Then we analysed the \emph{user interface} for this prototype, very similar to the previous one.

We move on to analyse the new Front-end \emph{controller}, which, although having the same functional model, has a different structural model, to account for the different structure of the documents indexed for this system. We analyse the new \ac{OoB} commands, as well as their parameters.

Like with the controller, the new \emph{conversational agent} is studied in this section, analysing the new file structure and control system.

To finish this chapter, we take a look at the Solr fields and schema built for this system, with the different analysers, accounting for the recovered documents being in English rather than Spanish. We also analyse the different queries that can be performed with this system. % Split in two part, the first one for architecture, the second in evaluation ? Or not...
\chapter{Evaluation}

\section{Overview}

\section{Requirements and Benchmark}

\section{Corpus tests}

\section{Comparison with other system}

\chapter{Conclusion and future work}
\label{chap:conclusion}


\begin{chapterintro}

In this chapter we will analyse the behaviour and performance of the system. We will also evaluate the accuracy of its responses, and the end user experience compared to a regular \ac{QA} system.
 
\end{chapterintro}

\cleardoublepage

\section{Conclusion}

For this project, we have developed multiple conversational agents that allowed users to interact with Linked Data Systems in natural language. We will now present the conclusions we have deduced from our work.

We started this document specifying several requirements and goals we aimed to achieve with the architecture and prototypes developed for this Master Thesis. It has been shown that with the technologies available at this point is possible build personal assistants using conversational agents and linked data to help final users find the information they are looking for.Building this systems comes from the appropriate knowledge of this technologies, and how they can communicate with each other is possible to build system. 

The Natural Language processing and understanding capabilities of ChatScript have allowed us to present an interface that users with low technical capabilities can use, effectively lowering the entry barrier for the fields we can adapt the system for. However, it is important to remark that Natural Language processing still needs to improve, as can be shown by the results of the evaluation. This is specially true for languages other than English, since most frameworks for natural language processing are aimed at English speaking users, and adapting them for different languages has proved it can be a daunting task if performed in full.

By developing two different prototypes for the proposed architecture, with different types of documents, we have shown that our system can be adapted to different fields with relatively low effort, although it has been shown than the first prototype was not ready for an environment formed by heterogeneus documents containing the required data.

The tests with the users have allowed us to check the hypotheses formulated at the start of this project, confirming several of them and being unable to support others.

Finally, the offering of our conversational agent as a web application, has proved a fundamental tool to spread the usage of our system, since it permits access form any device with an Internet connection.

\section{Achieved goals}

In chapter \ref{chapter:intro} we discussed the goals we wanted to achieve with this project. This can be summarize as follows:

\begin{enumerate}[label=(\roman*)]
 \item Develop a system that would be able to take a natural language request and provide an answer using a Linked Data System.
 \item Evaluate different natural language and linked data systems to use with our own.
 \item Study the implementation of the system in multiple fields of knowledge.
 \item Analyse the behaviour of the system, as well as the experience of the users using it.
\end{enumerate}

For the first goal, we have proposed and architecture that can fulfil that objective, as well as developed two prototypes proving it was possible to successfully follow that architecture while developing a system with a Natural Language Interface.

In order to achieve the first goal, we studied the systems for Natural Language processing available, analysing their capabilities and their weaknesses, focusing on the possibilities to use languages other than English for the processing. We also analysed multiple linked data systems, as well as linked open data systems, finally choosing the one we considered most appropriate for our system. Therefore, we consider the goal (ii) to have been reached.

The first prototype developed for our system was focused on providing assistance for an e-learning platform for the Java programming language, focusing on very simple and structured documents, therefore not presenting significant challenges to index. We then developed a system focused on finding information about a research group, with documents of different structures and formats, dealing with different topics. We achieved this by extending our first system into a new one capable of handling this new requirements. Accordingly, we consider goal (iii) to be achieved.

Finally, we proposed a methodology to test the system, and test whether our hypotheses about user experience and behaviour where correct. We perform an experiment comparing our system with a QA interface, with real users, and gathering data from their usage of the system. We also performed test to check the performance of the system under different loads, showing the final results in chapter \ref{chap:evaluation}. These tests achieved goal (iv).


\section{Future work}

Working on the systems developed for this Master Thesis, we have achieved an understanding of the possibilities of Natural Language Processing, as well as the benefits of the Semantic Web and the Linked Open Data initiative. With this knowledge, we can suggest future areas of study that can improve the functionality of the project and extend its goals.

\begin{enumerate}
 \item \emph{Spanish dictionaries}: Although it is possible to use the chatbot engine chosen for our prototypes, ChatScript, with Spanish, it has a distinct lack of support, nowhere near the capabilities for the English language. To improve this, it would be necessary to build Spanish dictionaries or translate the existing ones from English, as well as analyse the processing ChatScript does for English sentences and mimic it with Spanish, adapting it when necessary.
 \item \emph{Using SPARQL}: integrating a SPARQL query system would allow to easily adapt our system for multiple knowledge fields, as well as providing support for more complex interactions with the user, given the huge capabilities of the Linked Data.
 \item \emph{Automating the indexing process}: Although at this point the scrapping process is mostly automated and can be done without requiring human interactions, it is still necessary for a person to gather the scrapped data and indexing it in Solr. In order to make our system work in semi-real time, it would be necessary to fully automate the process, from launching the multiple scrappers to combine the information and adding it to the system, be it Solr or an SPARQL server.
 \item \emph{Extend the evaluation experiment to the second prototype}: The described experiment with the users was perform with the first prototype. Adapting the methodology to the second one, and gathering the users feedback may be useful for further improving the system.
 \item \emph{Adding audio interface}: Whilst we consider interacting with our system with natural language an improvement while comparing it with traditional interfaces, it could be improved further by including support for speech recognition and text-to-speech capabilities, simplifying user interactions, and greatly improving accessibility.
 \item \emph{Administrative interface}: Adding an interface to handle the different modules of the system, as well as checking their status and success rate in real time will allow administrators to easily manage the systems and respond to any possible problem that may appear.
\end{enumerate}

In conclusion, we have developed a robust and functional system, but it can still be greatly improved with new modules, enhancing the user experience. As new ideas and new developers come forward, they could take this system as a base to develop better and more powerful personal assistants.

%\appendix
\chapter{Installation Manual}

This chapter aims to provide a general walk-trough on how to deploy our prototypes bot on a fresh system. It assumes basic knowledge on both the system and the tools used, and the required tools.

\section{Order of deployment}

Our system is composed of several submodules, and it is reccomended that they are started in a specific order:
\begin{enumerate}
 \item Chatscript
 \item Apache Solr
 \item Front-end controller
 \item Front-end client
\end{enumerate}

\section{Chatscript}

Before being able to launch ChatScript, if you are using a 64 bit system, you need to install several 32-bit libraries, as shown in \ref{listing:cslibs}:

\begin{center} 
  \begin{lstlisting}[language=bash, captionpos=b, caption=Libraries for ChatScript, label=listing:cslibs]
  :~/$ sudo dpkg --add-architecture i386
  :~/$ sudo apt-get update
  :~/$ sudo apt-get install lib32gcc1 lib32stdc++6
  \end{lstlisting}
\end{center}


For the first launch of Chatscript, you need to compile the corpus. Therefore, you have to launch chatscript on localmode first, issue the build commands, and then launch it as a server. The commands are shown in \ref{listing:csbuild} for the build, and, once de process is complete, exit the local interface with ``:quit'', and launch it as a server, as shown in \ref{listing:csserver}:

\begin{center} 
  \begin{lstlisting}[language=bash, captionpos=b, caption=ChatScript build command, label=listing:csbuild]
  :~/calista-bot/ChatScript$ ./LinuxChatSscript64 local
  [...]
  Enter user name: username
  username: > :build Dent
  [...]
  username: > :quit
  \end{lstlisting}
\end{center}

Keep in mind that, even though we refer to our bot as ``Duke'', the files for this project are the ones from the ``Dent'' folder, since the existing ``Duke'' files are related to a previous iteration and where keep independent.

\begin{center} 
  \begin{lstlisting}[language=bash, captionpos=b, caption=ChatScript Server running, label=listing:csserver]
  :~/calista-bot/ChatScript$ ./LinuxChatScript64 port=1025
arg[1] = port=1025
evserver child fork requests: 0
  evserver: running
Server EVSERVER ChatScript Version 5.1  64 bit LINUX compiled Feb  1 2015 23:00:53
EVSERVER ChatScript Version 5.1  64 bit LINUX compiled Feb  1 2015 23:00:53
Params:   dict:720895 fact:800000 text:35000kb hash:50000 
          buffer:22x80kb cache:1x50kb userfacts:100
WordNet: dict=190271  fact=84865  stext=12742900 Jan31'15-15:26:50
Build0:  dict=68775  fact=130830  dtext=1162676 stext=0 Feb01'15-15:07:58 0.txt
Build1:  dict=8  fact=7  dtext=356 stext=3664 Jun27'15-22:42:15 Niko.txt
Error(2) opening LIVEDATA/ENGLISH
Used 49MB: dict 259,059 (24869kb) hashdepth 17/3 fact 215,702 (8628kb) text 13910kb
           buffer (1760kb) cache (50kb) POS: 0 (0kb)
Free 44MB: dict 461,836 hash 1136 fact 584,298 text 21,089KB 

    Dictionary building disabled.
    Postgres disabled.



======== Began EV server 5.1 compiled Feb  1 2015 23:00:53 on port 1025 at Tue Jun 30 23:41:37 2015 serverlog:1 userlog: 0


======== Began EV server 5.1 compiled Feb  1 2015 23:00:53 on port 1025 at Tue Jun 30 23:41:37 2015 serverlog:1 userlog: 0
evserver: parent ready (pid = 12435), fork=0
EVServer ready: LOGS/serverlog1025.txt

  \end{lstlisting}
\end{center}

Keep in mind it will bind to the 1025 port.

\section{Front-end controller}

The controller need two libraries to be installed with pip. It is recommended to do it in a virtualenvironment, as shown in \ref{listing:askbotlibs}.

\begin{center} 
  \begin{lstlisting}[language=bash, captionpos=b, caption=Libraries for the front end controller, label=listing:askbotlibs]
  :~/$ mkvirtualenv askbot
  (askbot):~/$ pip install websocket-client unidecode
  \end{lstlisting}
\end{center}

The askbot controller binds, by default, to the 4242 port, although it can be changed in the askbot.py file. Once the changes have been made, it can just be run with python as in \ref{listing:askbotrun}
\begin{center} 
  \begin{lstlisting}[language=bash, captionpos=b, caption=Running the front end controller, label=listing:askbotrun]
  (askbot):~/calista-bot/FE-Controller$ python askbot.py
  \end{lstlisting}
\end{center}

\subsection{Using apache mod\_wsgi}

Using the provided askbot.wsgi, the system can be run using Apache and mod\_wsgi. After installing and enabling the mod (in Debian systems, see listing \ref{listing:wsgiinstall}, set the virtualenvironment in the configuration files, as well as the appropriate Aliases, as shown in \ref{listing:apachewsgi}
\begin{center} 
  \begin{lstlisting}[language=bash, captionpos=b, caption=Installing mod\_wsgi, label=listing:wsgiinstall]
  (askbot):~/$ sudo apt-get install libapache2-mod-wsgi
  (askbot):~/$ sudo a2enmod wsgi
  (askbot):~/$ sudo apache2ctl restart
  \end{lstlisting}
\end{center}

\begin{center} 
  \begin{lstlisting}[language={}, captionpos=b, caption=Apache WSGI configuration, label=listing:apachewsgi]
    # Run with the virtual environment
    WSGIDaemonProcess talkbot python-path=/path/to/virtualenvsfolder/VEnvs/talkbot/lib/python2.7/site-packages
    
    # The path for the .wsgi
    WSGIScriptAlias /AskBot /path/to/calista-bot/FE-Controller/askbot.wsgi
    <Location /AskBot>
        WSGIProcessGroup talkbot
        Order deny,allow
        Allow from all
    </Location>
  \end{lstlisting}
\end{center}

\section{Front-end client}

The client for the bot is a web application. You just need to place the Ask-client files on a webserver (like Apache or Nginx) html folder, and edit the index.html file, so the action attribute in the form points to the host and port from the previous step, or to the Apache alias is set up with mos\_wsgi.

\section{Apache Solr}

Apache Solr can be run in a Tomcat installation. You will need to indicate the solr.home in a context specific for Solr, where the schemas and the Solr libraries are located. An example context is shown in \ref{listing:tomcatsolr}. This file should be stored inside the ``conf/Catalina/localhost'' folder, and with the same base name as the .war. For example, for an application solr.war, the appropriate context file will be solr.xml
\begin{center} 
  \begin{lstlisting}[language=XML, captionpos=b, caption=Tomcat context for solr, label=listing:tomcatsolr]
  <Context docBase="/Path/to/tomcat/Tomcat/webapps/solr.war" debug="0" crossContext="true">
    <!-- The path to the folder where the solr installation is located 
         Production environments should avoid using the example Solr instance -->
    <Environment name="solr/home" type="java.lang.String" value="/home/amardomingo/PFC/Solr/example/solr" override="true"/>
    
    <!-- Allow access to Solr only from localhost -->
    <Valve className="org.apache.catalina.valves.RemoteHostValve" allow="localhost"/>
  </Context>
  \end{lstlisting}
\end{center}

\include{Appendix-scrappy}
\chapter{Solr Uploader}

The complete RDF scrapped for this document is avaliable in \url{http://github.com/gsi-upm/calista-bot}, since the files are too long to be included here. We will show, however, the uploader developed to index the files in solr:

\section{Command syntaxes}

The uploader can handle rdf and json files, and takes the arguments explained in table \ref{tab:uploaderparams} . And example usage is shown in listing \ref{listing:rdfupload}

\begin{table}
  \centering
  \begin{tabular*}{0.9\textwidth}{| c | c | c | p{0.5\textwidth} |}
    \hhline{|-|-|-|-|}
    \textbf{Argument} & \textbf{Short argument} & \textbf{Parameter}& \textbf{Explanation} \\ \hhline{|=|=|=|=|}
    --help & -h &  & Shows the help\\ \hhline{|-|-|-|-|} 
    --url & -u & URL & Sets URL as the Solr endpoint \\ \hhline{|-|-|-|-|} 
    --core & -c & CORE & Uses CORE as the Solr core to upload the data to\\ \hhline{|-|-|-|-|} 
    --data & -d & JSONFILE & Reads the json data in the JSONFILE file to upload it to solr\\ \hhline{|-|-|-|-|} 
    --rdf & -r & RDFFILE & Reads the rdf data in the RDFFILE file to upload it to solr \\ \hhline{|-|-|-|-|} 
    --verbose & -v & & Prints debug information \\ \hhline{|-|-|-|-|} 
    --empty & -e & & Clears all the data in solr before uploading any document \\ \hhline{|-|-|-|-|} 
    --output & -o & & Logfile to output to \\ \hhline{|-|-|-|-|} 
    \end{tabular*}
  \caption{Parameters for the uploader to Solr.}
  \label{tab:uploaderparams}
\end{table}


\begin{center} 
  \begin{lstlisting}[language=bash, captionpos=b, caption=Example command to upload the vademecum data into Solr, label=listing:rdfupload]
  :~/calista-bot/RDF$ ./uploader.py -u http://localhost:8080/solr -c elearning -r vademecum.rdf -e
  \end{lstlisting}
\end{center}

\section{Uploader code}

\begin{center}
  \lstinputlisting[language=python, captionpos=b, stringstyle=\color{Strings}, caption=Uploader to index the scrapped data in solr, label=listing:vademecumrdf]{code/anex/uploader.py}
\end{center}


\phantomsection
\addcontentsline{toc}{chapter}{Bibliography}
\bibliographystyle{ieeetr}
{
\small
\bibliography{biblio/references}
}
\cleardoublepage
\end{document}
