\chapter{Solr Uploader}

The complete RDF scrapped for this document is avaliable in \url{http://github.com/gsi-upm/calista-bot}, since the files are too long to be included here. We will show, however, the uploader developed to index the files in solr:

\section{Command syntaxes}

The uploader can handle rdf and json files, and takes the arguments explained in table \ref{tab:uploaderparams} . And example usage is shown in listing \ref{listing:rdfupload}

\begin{table}
  \centering
  \begin{tabular*}{0.9\textwidth}{| c | c | c | p{0.5\textwidth} |}
    \hhline{|-|-|-|-|}
    \textbf{Argument} & \textbf{Short argument} & \textbf{Parameter}& \textbf{Explanation} \\ \hhline{|=|=|=|=|}
    --help & -h &  & Shows the help\\ \hhline{|-|-|-|-|} 
    --url & -u & URL & Sets URL as the Solr endpoint \\ \hhline{|-|-|-|-|} 
    --core & -c & CORE & Uses CORE as the Solr core to upload the data to\\ \hhline{|-|-|-|-|} 
    --data & -d & JSONFILE & Reads the json data in the JSONFILE file to upload it to solr\\ \hhline{|-|-|-|-|} 
    --rdf & -r & RDFFILE & Reads the rdf data in the RDFFILE file to upload it to solr \\ \hhline{|-|-|-|-|} 
    --verbose & -v & & Prints debug information \\ \hhline{|-|-|-|-|} 
    --empty & -e & & Clears all the data in solr before uploading any document \\ \hhline{|-|-|-|-|} 
    --output & -o & & Logfile to output to \\ \hhline{|-|-|-|-|} 
    \end{tabular*}
  \caption{Parameters for the uploader to Solr.}
  \label{tab:uploaderparams}
\end{table}


\begin{center} 
  \begin{lstlisting}[language=bash, captionpos=b, caption=Example command to upload the vademecum data into Solr, label=listing:rdfupload]
  :~/calista-bot/RDF$ ./uploader.py -u http://localhost:8080/solr -c elearning -r vademecum.rdf -e
  \end{lstlisting}
\end{center}

\section{Uploader code}

\begin{center}
  \lstinputlisting[language=python, captionpos=b, stringstyle=\color{Strings}, caption=Uploader to index the scrapped data in solr, label=listing:vademecumrdf]{code/anex/uploader.py}
\end{center}
